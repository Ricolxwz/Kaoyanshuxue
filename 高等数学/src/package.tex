\PassOptionsToPackage{AutoFakeBold}{xeCJK}
\PassOptionsToPackage{quiet}{fontspec}
% 中文环境
\usepackage[UTF8]{ctex}
% 合并单元格
\usepackage{multirow}
% 双栏
\usepackage{multicol}
% 设置单元格格式
\usepackage{makecell}
% 表格
\usepackage{tabularx}
\newcolumntype{Z}{>{\centering\let\newline\\\arraybackslash\hspace{0pt}}X}
% 首行缩进
\usepackage{indentfirst}
\setlength{\parindent}{2em}
% 表格格式
\usepackage{floatrow}
\floatsetup[table]{font={large, tt}}
% \bm{}
\usepackage{bm}
% 浮动体
\usepackage{float}
% 矩阵
\usepackage{array}
% 三线表格
\usepackage{threeparttable}
% 数学包
\usepackage{amsmath}
% 图片
\usepackage{graphicx}
% 自定义颜色
\usepackage{xcolor}
% 带圈的数字
\usepackage{pifont}
% 超链接
\usepackage{hyperref}
\hypersetup{hypertex=true,
colorlinks=true,
linkcolor=blue,
anchorcolor=blue,
citecolor=blue}
% tikz
\usepackage{tikz}
\usetikzlibrary{external}
\tikzset{
external/system call={
xelatex \tikzexternalcheckshellescape
-halt-on-error -interaction=batchmode --shell-escape
-jobname "\image" "\texsource"}}
\tikzexternalize[prefix=tikz/]
% 绘制函数
\usepackage{pgfplots}
\pgfplotsset{compat=newest}
% 框
\usepackage{tcolorbox}
% 格式化图表标题
\usepackage{subcaption}
% 自定义数学公式
\DeclareMathOperator{\arccot}{arccot}
% 大括号
\usepackage{scalerel}
% 页面设置
\usepackage{geometry}
\geometry{scale=0.9}
% 代码
\usepackage{listings}
\lstset{
basicstyle=\ttfamily,
keywordstyle=\bfseries,
commentstyle=\rmfamily\itshape,
stringstyle=\ttfamily,
flexiblecolumns,
numbers=left,
showspaces=false,
numberstyle=\zihao{-5}\ttfamily,
showstringspaces=false,
captionpos=t,
frame=lrtb,
}
% 数学公式自定义字体
\usepackage{unicode-math}