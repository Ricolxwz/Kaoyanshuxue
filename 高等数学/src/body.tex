\chapter{预备知识}
\section{基础知识}
\subsection{函数的概念和特性}
\subsubsection{函数}
设$ x $与$ y $是两个变量, $ D $是一个给定的数集, 若对于每一个$ x\in D $, 按照一定的法则$ f $, 有一个唯一确定的$ y $与之对应, 则称$ y $为$ x $的函数, 记为$ y=f(x) $, 称$ x $为自变量, $ y $为因变量, $ D $为定义域.
\subsubsection{反函数}
设函数$ y=f(x) $的定义域为$ D $, 值域为$ R $, 若对于每一个$ y\in R $, 必存在唯一的$ x\in D $使得$ y=f(x) $成立, 则由此定义了一个新的函数$ x=\varphi(y) $, 称这个函数是$ y=f(x) $的反函数, 一般记作$ x=f^{-1}(y) $, 它的定义域为$ R $, 值域为$ D $. \par
\begin{tcolorbox}
\begin{enumerate}
\item 严格单调的函数一定有反函数(严格单调函数不一定是反函数, 如某些分段函数)
\item $ x=f^{-1}(y) $和$ y=f(x) $是同一个函数, 只有写成$ y=f^{-1}(x) $, 图像才关于$ y=x $对称
\end{enumerate}
\end{tcolorbox}
\subsubsection{复合函数}
函数$ u=g(x) $在$ x\in D $上有定义, 函数$ y=f(u) $在$ u\in D_{1} $上有定义, 且$ g(D)\subset D_{1} $, 则称$ y=f(g(x)) $为复合函数, 定义域为$ D $, $ u $为中间变量.
\subsubsection{函数的四种特性和重要结论}
\begin{enumerate}
\item 有界性\par
设$ f(x) $的定义域为$ D $, 数集$ I\subset D $. 若存在某个正数$ M $, 使得对于任一$ x\in I $, 有$ |f(x)|\le M $成立, 则称$ f(x) $在$ I $上有界. 如果这样的$ M $不存在, 则称$ f(x) $在$ I $上无上界.
\item 单调性\par
设$ f(x) $的定义域为$ D $, 区间$ I\subset D $, 如果对于区间上的任一两点$ x_{1},x_{2} $, 当$ x_{1}<x_{2} $的时候有$ f(x_{1})<f(x_{2}) $成立, 则称$ f(x) $在$ I $上单调增加. 反之如果$ f(x_{1})>f(x_{2}) $成立, 则称$ f(x) $在$ I $上单调减少.
\item 奇偶性\par
设$ f(x) $的定义域$ D $关于原点对称. 如果对于任一$ x\in D $, 恒有$ f(x)=f(-x) $, 则称$ f(x) $为偶函数. 如果对于任一$ x\in D $, 恒有$ f(x)=-f(-x) $, 则称$ f(x) $为奇函数. 偶函数的图像关于$ y $轴对称, 奇函数的图像关于原点对称.
\begin{tcolorbox}
\begin{enumerate}
\item 奇函数在$ 0 $点有定义则$ f(0)=0 $
\item 偶函数当$ f'(0) $存在时则$ f'(0)=0 $
\item 函数$ f(x) $和$ -f(x) $关于$ x $轴对称, 函数$ f(x) $和$ f(-x) $关于$ y $轴对称, 函数$ y(x) $和$ -y(-x) $关于原点对称
\item 函数$ f(x) $关于$ x=T $对称$ \Leftrightarrow f(x+T)=f(T-x) $
\end{enumerate}
\end{tcolorbox}
\item 周期性\par
设$ f(x) $的定义域为$ D $, 若存在一个正数$ T $, 使得对于任一$ x\in D $, 有$ x\pm T\in D $, 且$ f(x+T)=f(x) $. 则称$ f(x) $为周期函数, $ T $称为$ f(x) $的周期.
\item 重要结论
\begin{enumerate}
\item 函数和其导函数
\subitem 偶函数的导函数是奇函数
\subitem 奇函数的导函数是偶函数
\subitem 周期函数的周期和其导函数的周期相同
\item 函数和其原函数
\subitem 连续的奇函数的原函数是偶函数
\subitem 连续的偶函数的原函数只有一个是奇函数
\subitem 连续的周期函数和其原函数的周期相同
\item 若$ f(x) $在$ (a,b) $内可导且$ f'(x) $有界, 则$ f(x) $在$ (a,b) $内有界
\end{enumerate}
\end{enumerate}
\subsection{函数的图像}
\subsubsection{直角坐标系}
\begin{enumerate}
\item 常见图像
\begin{enumerate}
\item 基本初等函数与初等函数
\begin{enumerate}
\item 常数函数\par
$ y=C $, $ C $为常数, 图形为平行于$ x $轴的水平直线.
\item 幂函数\par
$ y=x^{\mu} $\ ($ \mu $是实数)
\begin{tcolorbox}
\begin{enumerate}
\item 见到$ \sqrt{u},\sqrt[3]{u} $, 用$ u $来研究最值
\item 见到$ |u| $时, 用$ u^{2} $来研究最值
\item 见到$ u_{1}u_{2}u_{3} $时, 用$ ln(u_{1}u_{2}u_{3})=lnu_{1}+lnu_{2}+lnu_{3} $来研究最值
\item 见到$ \frac{1}{u} $时, 用$ u $来研究最值
\end{enumerate}
\end{tcolorbox}
\item 指数函数
$ y=a^{x} $\ ($ a>0,a\neq 1 $)
\item 对数函数
$ y=log_{a}x $\ ($ a>0,a\neq 1 $)
\begin{tcolorbox}
常用公式: $ x=e^{lnx}\ (x>0), u^{v}=e^{lnu^{v}}=e^{vlnu}\ (u>0) $
\end{tcolorbox}
\item 三角函数
\begin{enumerate}
\item 正弦函数和余弦函数\par
正弦函数$ y=\sin x $, 余弦函数$ y=\cos x $.
\item 正切函数和余切函数\par
正切函数$ y=\tan x $, 余切函数$ y=\cot x $.\par
\begin{figure}[htp]
\centering
\begin{subfigure}{.475\linewidth}
\centering
\begin{tikzpicture}[
]
\begin{axis}[
width=\linewidth,
axis lines=middle,
xmin=-4.3,
xmax=4.3,
ymin=-4.5,
ymax=4.5,
xlabel=$ x $,
ylabel=$ y $,
xlabel=$ x $,
xlabel style={below},
ylabel=$ y $,
ylabel style={left},
xtick={-pi/2,pi/2},
xticklabels={$ -\frac{\pi}{2} $,$ \frac{\pi}{2} $},
xticklabel style={left,yshift=-0.5em},
ytick={1},
yticklabels={$ 1 $},
extra y ticks={-1},
extra y tick labels={$ -1 $},
extra y tick style={tick label style={right,yshift=-1em}},
]
\addplot [black,samples=1000]{tan(deg(x))};
\end{axis}
\end{tikzpicture}
\caption{正切函数图像}
\end{subfigure}
\hspace{.1em}
\begin{subfigure}{.475\linewidth}
\centering
\begin{tikzpicture}[
]
\begin{axis}[
width=\linewidth,
axis lines=middle,
xmin=-4.3,
xmax=4.3,
ymin=-4.5,
ymax=4.5,
xlabel=$ x $,
xlabel style={below},
ylabel=$ y $,
ylabel style={left},
xtick={-pi/2,pi/2},
xticklabels={$ -\frac{\pi}{2} $,$ \frac{\pi}{2} $},
xticklabel style={left,yshift=-0.5em},
ytick={1},
yticklabels={$ 1 $},
extra y ticks={-1},
extra y tick labels={$ -1 $},
extra y tick style={tick label style={right,yshift=-1em}},
]
\addplot [black,samples=1000]{1/tan(deg(x))};
\end{axis}
\end{tikzpicture}
\caption{余切函数图像}
\end{subfigure}
\end{figure}
\item 正割函数和余割函数\par
正割函数$ y=\sec x $, 余割函数$ y=\csc x $.
\begin{figure}[htp]
\centering
\begin{subfigure}{.475\linewidth}
\centering
\begin{tikzpicture}[
]
\begin{axis}[
width=\linewidth,
axis lines=middle,
xmin=-4.3,
xmax=4.3,
ymin=-4.5,
ymax=4.5,
xlabel=$ x $,
ylabel=$ y $,
xtick={-pi/2,pi/2},
xticklabels={$ -\frac{\pi}{2} $,$ \frac{\pi}{2} $},
xticklabel style={left,yshift=-0.5em},
ytick={-1,1},
yticklabels={$ -1 $,$ 1 $},
yticklabel style={below,xshift=-1em,yshift=0.4em},
]
\addplot [black,samples=1000]{1/cos(deg(x))};
\end{axis}
\end{tikzpicture}
\caption{正割函数图像}
\end{subfigure}
\begin{subfigure}{.475\linewidth}
\begin{tikzpicture}[
]
\begin{axis}[
width=\linewidth,
axis lines=middle,
xmin=-4.3,
xmax=4.3,
ymin=-4.5,
ymax=4.5,
xlabel=$ x $,
ylabel=$ y $,
ylabel style={left},
xtick={pi/2},
xticklabels={$ \frac{\pi}{2} $},
extra x ticks={-pi/2},
extra x tick labels={$ -\frac{\pi}{2} $},
extra x tick style={tick label style={above}},
ytick={1},
yticklabels={$ 1 $},
extra y ticks={-1},
extra y tick labels={$ -1 $},
extra y tick style={tick label style={right,yshift=-0.5em}},
]
\addplot [black,samples=1000]{1/sin(deg(x))};
\end{axis}
\end{tikzpicture}
\caption{余割函数图像}
\end{subfigure}
\end{figure}
\end{enumerate}
\item 反三角函数
\begin{enumerate}
\item 反正弦函数和反余弦函数\par
反正弦函数$ y=\arcsin x $, 反余弦函数$ y=\arccos x $.
\begin{figure}[htp]
\centering
\begin{subfigure}{.475\linewidth}
\centering
\begin{tikzpicture}[
]
\begin{axis}[
width=\linewidth,
axis lines=middle,
domain=-1:1,
xmin=-2,
xmax=2,
ymin=-2,
ymax=2,
xlabel=$ x $,
ylabel=$ y $,
xtick={-1,1},
ytick={-pi/2,pi/2},
yticklabels={$ -\frac{\pi}{2} $, $ \frac{\pi}{2} $},
]
\addplot [black,samples=1000]{asin(x)/180*pi};
\end{axis}
\end{tikzpicture}
\caption{反正弦函数图像}
\end{subfigure}
\begin{subfigure}{.475\linewidth}
\begin{tikzpicture}[
]
\begin{axis}[
width=\linewidth,
axis lines=middle,
domain=-1:1,
xmin=-2,
xmax=2,
ymin=-0.5,
ymax=4,
xlabel=$ x $,
ylabel=$ y $,
ytick={pi/2,pi},
yticklabels={$ \frac{\pi}{2} $, $ \pi $},
xtick={-1,1},
xticklabels={$ -1 $,$ 1 $},
]
\addplot [black,samples=1000]{acos(x)/180*pi};
\end{axis}
\end{tikzpicture}
\caption{反余弦函数图像}
\end{subfigure}
\end{figure}
\item 反正切函数和反余切函数\par
反正切函数$ y=\arctan x $, 反余切函数$ y=\arccot x $
\begin{figure}[htp]
\centering
\begin{subfigure}{.475\linewidth}
\centering
\begin{tikzpicture}[
]
\begin{axis}[
width=\linewidth,
axis lines=middle,
xmin=-2,
xmax=2,
ymin=-2,
ymax=2,
xlabel=$ x $,
ylabel=$ y $,
xtick={\empty},
ytick={-pi/2,pi/2},
yticklabels={$ -\frac{\pi}{2} $, $ \frac{\pi}{2} $},
]
\addplot [black,samples=1000]{atan(x)/180*pi};
\end{axis}
\end{tikzpicture}
\caption{反正切函数图像}
\end{subfigure}
\begin{subfigure}{.475\linewidth}
\begin{tikzpicture}[
]
\begin{axis}[
width=\linewidth,
axis lines=middle,
xmin=-2,
xmax=2,
ymin=-0.5,
ymax=4,
xlabel=$ x $,
ylabel=$ y $,
ytick={pi/2,pi},
yticklabels={$ \frac{\pi}{2} $, $ \pi $},
xtick={\empty},
]
\addplot [black,samples=1000]{(pi/2)-(atan(x))/180*pi};
\end{axis}
\end{tikzpicture}
\caption{反余切函数图像}
\end{subfigure}
\end{figure}
\end{enumerate}
\item 初等函数\par
由基本初等函数经过有限次的四则运算, 以及有限次的复合所构成的可以用一个式子表示的函数称为初等函数.
\end{enumerate}
\item 分段函数\par
在自变量的不同范围中, 对应法则不同式子来表示的函数称为分段函数. 一般来说它不是初等函数.
\begin{enumerate}
\item 绝对值函数\par
\begin{equation*}
y=|x|=
\scaleleftright[6pt]{\biggl\{}{
\begin{aligned}
& x,\ x\ge 0 \\
& -x,\ x<0
\end{aligned}
}{.}
\end{equation*}
\item 符号函数\par
\begin{equation*}
y=\text{sgn}\ x=
\scaleleftright[6pt]{\biggl\{}{
\begin{aligned}
& 1,\ x>0 \\
& 0,\ x=0 \\
& -1,\ x<0
\end{aligned}
}{.}
\end{equation*}
\item 取整函数\par
$ y=[x] $, 设$ x $为任一实数, 不超过$ x $的最大整数称为$ x $的整数部分, 记作$ [x] $.
\end{enumerate}
\end{enumerate}
\item 图像变换
\begin{enumerate}
\item 平移变换
\begin{enumerate}
\item 将函数$ y=f(x) $沿$ x $轴向左平移$ x_{0}\ (x_{0}>0) $个单位长度, 得到函数$ f(x+x_{0}) $的图像; 将函数$ y=f(x) $沿$ x $轴向右平移$ x_{0}\ (x_{0}>0) $个单位长度, 得到函数$ f(x-x_{0}) $的图像
\item 将函数$ y=f(x) $沿$ y $轴向上平移$ y_{0}\ (y_{0}>0) $个单位长度, 得到函数$ f(x)+y_{0} $的图像; 将函数$ y=f(x) $沿$ y $轴向下平移$ y_{0}\ (y_{0}>0) $个单位长度, 得到函数$ f(x)-y_{0} $的图像
\end{enumerate}
\item 对称变换

\end{enumerate}
\end{enumerate}



















