\documentclass[oneside, onecolumn]{ctexbook}
\usepackage[UTF8]{ctex}
\usepackage{indentfirst}
\usepackage{bm}
\usepackage{float}
\usepackage{amsmath}
\setlength{\parindent}{2em}
\begin{document}
\chapter{特征值和特征向量}
\section{基础知识}
\subsection{特征值和特征向量}
\subsubsection{定义}
设$ \bm{A} $为$ n $阶矩阵, $ \lambda $是一个数, 若存在一个非零的$ n $维向量$ \bm{\xi} $, 使得$ \bm{A}\bm{\xi}=\lambda \bm{\xi} $, 则称$ \bm{\xi} $为$ \bm{A} $的特征向量, $ \lambda $为$ \bm{A} $的特征值. 
\par 上式可以化简成$ \left| \lambda \bm{E}-\bm{A}\right|\bm{\xi}=\bm{0} $, $ \left| \lambda \bm{E}-\bm{A}\right| $被称为特征多项式, $ \lambda \bm{E}-\bm{A} $称为特征矩阵.
\subsubsection{性质}
\begin{enumerate}
	\item 特征值的性质
	\begin{enumerate}
		\item $ \sum_{i=1}^{n}\lambda_{i}=\sum_{i=1}^{n}a_{ii}=tr(\bm{A}) $
		\item $ \prod_{i=1}^{n}\lambda_{i}=\left| \bm{A}\right| $
		\end{enumerate}
	\item 特征向量的性质
	\begin{enumerate}
		\item $ k $重特征值$ \Lambda $至多只有$ k $个线性无关的向量
		\item 若$ \bm{\xi_1} $, $ \bm{\xi_2} $是$ \bm{A} $的属于不同特征值的特征的特征向量, 则$ \bm{\xi_1}, \bm{\xi_2} $线性无关
		\item 若$ \bm{\xi_1}, \bm{\xi_2} $是$ \bm{A} $的属于同一特征值$ \lambda $的特征向量, 则$ k_1\bm{\xi_1} + k_1\bm{\xi_2} $仍然是$ \bm{A} $的属于特征值$ \lambda $的特征向量
	\end{enumerate}
\end{enumerate}
\subsection{矩阵的相似}
\subsubsection{定义}
设$ \bm{A} $和$ \bm{B} $为两个$ n $阶方阵, 若存在$ n $阶可逆矩阵$ \bm{P} $, 使得$ \bm{P}^{-1}\bm{A}\bm{P}=\bm{B} $成立, 则称$ \bm{A} $相似于$ \bm{B} $, 记成$ \bm{A}\sim \bm{B} $.
\subsubsection{性质}
\begin{enumerate}
	\item \begin{itemize}
		\item \textbf{反身性}: $ \bm{A}\sim \bm{A} $
		\item \textbf{对称性}: $ \bm{A}\sim \bm{B}\Rightarrow \bm{B}\sim \bm{A} $
		\item \textbf{传递性}: $ \bm{A}\sim \bm{B}, \bm{B}\sim \bm{C}\Rightarrow \bm{A}\sim \bm{C} $
	\end{itemize}
    \item 若$ \bm{A}\sim \bm{B} $, 则有\begin{itemize}
    	\item $ r(\bm{A})=r(\bm{B}) $
    	\item $ \left|\bm{A}\right|=\left|\bm{B}\right| $
    	\item $ \bm{A}, \bm{B} $具有相同的特征值
    	\item $ \bm{A}, \bm{B} $特征多项式的值相同
    \end{itemize}
    \item 若$ \bm{A}\sim \bm{B} $, 则有\begin{itemize}
    	\item $ f(\bm{A})\sim f(\bm{B}) $
    	\item $ \bm{A}^{T}\sim \bm{B}^{T} $
    	\item $ \bm{A} $可逆, $ \bm{A}^{*}\sim \bm{B}^{*} $
    	\item $ \bm{A} $可逆, $ \bm{A}^{-1}\sim \bm{B}^{-1} $
    \end{itemize}
\end{enumerate}
\subsection{矩阵的相似对角化}
\subsubsection{定义}
设$ n $阶矩阵$ \bm{A} $, 存在$ n $阶可逆矩阵$ \bm{P} $, 使得$ \bm{P}^{-1}\bm{A}\bm{P}=\bm{\Lambda} $, 则$ \bm{A}\sim \bm{\Lambda} $, $ \bm{\Lambda} $是$ \bm{A} $的相似标准形.\par 
\[ \bm{P}=\left[\bm{\xi_1}, \bm{\xi_2},... \bm{\xi_n}\right], \bm{\Lambda} =\begin{bmatrix}
	\lambda_1 &  &  &  \\
	& \lambda_2 &  &  \\
	&  & \ddots &  \\
	&  &  & \lambda_n 
\end{bmatrix}
 \]
\subsubsection{条件}
\begin{enumerate}
	\item $ n $阶矩阵$ \bm{A} $可以相似对角化$ \Leftrightarrow $$ \bm{A} $有$ n $个线性无关的特征向量($ \left| \bm{P}\right| = 0$)
	\item $ n $阶矩阵$ \bm{A} $可以相似对角化$ \Leftrightarrow $$ \bm{A} $对应于每个$ k_i $重特征值都有$ k_i $个线性无关的特征向量($ n $重特征值对应的解空间是$ n $维)
	\item $ n $阶矩阵$ \bm{A} $有$ n $个不同特征值$ \Rightarrow $$ \bm{A} $可以相似对角化(由特征向量的性质3可以推出)
	\item $ n $阶矩阵$ \bm{A} $为实对称矩阵$ \Rightarrow $$ \bm{A} $可以相似对角化
\end{enumerate}\par
上述总共两个充要条件, 两个充分条件.
\subsection{实对称矩阵}
\subsubsection{定义}
若$ \bm{A}^T = \bm{A} $, 则$ \bm{A} $为是对称矩阵, 如果在此基础上$ \bm{A} $的元素都是实数, 则$ \bm{A} $是实对称矩阵.
\subsubsection{性质}
\begin{enumerate}
	\item 实对称矩阵$ \bm{A} $的属于不同特征值的特征向量相互正交
	\item 实对称矩阵$ \bm{A} $必相似于对角矩阵, 且存在正交矩阵$ \bm{Q} $, 使得$ \bm{Q}^{-1}\bm{A}\bm{Q}=\bm{Q}^{T}\bm{A}\bm{Q}=\bm{\Lambda}$,  
\end{enumerate}
\section{习题}
\subsection{特征值和特征向量}
\subsubsection{求具体型矩阵的特征值和特征向量}
\begin{enumerate}
	\item 用特征方程$ \left|\lambda \bm{E}-\bm{A}\right|=0 $求出$ \lambda $, 可以使用试根法对$ \lambda $的高次方程进行求解
	\item 用求得的$ \lambda $解齐次线性方程组$ (\lambda \bm{E}-\bm{A})\bm{\xi}=\bm{0} $, 求出特征向量
\end{enumerate}
\subsubsection{求解抽象型矩阵的特征值和特征向量}
\begin{table}[H]
 	\centering
 	\begin{tabular}{|c|ccccccc|}
 		\hline
 		\textbf{矩阵} & $ \bm{A} $ & $ k\bm{A} $ & $ \bm{A}^{k} $ & $ f(\bm{A}) $ & $ \bm{A}^{-1} $ & $ \bm{A}^{*} $ & $ \bm{P}^{-1}\bm{A}\bm{P} $ \\ \hline
 		\textbf{特征值} & $ \lambda $ & $ k\lambda $ & $ \lambda^{k} $ & $ f(\lambda) $ & $ \lambda^{-1} $ & $ \frac{\left|\bm{A}\right|}{\lambda} $ & $ \lambda $\\ \hline
 		\textbf{特征向量} & $ \bm{\xi} $ & $ \bm{\xi} $ & $ \bm{\xi} $ & $ \bm{\xi} $ & $ \bm{\xi} $ & $ \bm{\xi} $ & $ \bm{P}^{-1}\bm{\xi} $\\ 
 		\hline
 	\end{tabular}
\end{table}\par
$ f(x) $为多项式, 若矩阵$ \bm{A} $满足$ f(\bm{A})=\bm{0}\Rightarrow f(\lambda)=0 $.
\subsection{实对称矩阵}
\subsubsection{求正交矩阵Q}
\begin{enumerate}
	\item 求$ \bm{A} $的$ \lambda $与$ \bm{\xi} $
	\item $ \bm{\xi_1}, \bm{\xi_{2}},... ,\bm{\xi_{n}} $施密特正交化, 单位化至$ \bm{\eta_{1}}, \bm{\eta_{2}},... ,\bm{\eta_{n}} $
	\item 令$ Q=(\bm{\eta_1}, \bm{\eta_2},... ,\bm{\eta_{n}}) $
\end{enumerate}
\par 不同的特征值$ \lambda_{i} $对应的特征矩阵$ \bm{\xi_{i}} $之间是正交的.
\par 施密特正交化:$ \beta_{1}=\alpha_{1}, 
\beta_{2}=\alpha_{2}-\frac{(\alpha_{2}, \beta_{1})}{(\beta_{1}, \beta_{1})}\beta_{1} $.
\par 单位化: $ \eta_{1}=\frac{\beta_{1}}{||\beta_{1}||} $.
\paragraph{总结} 
\begin{enumerate}
	\item 普通矩阵$ \bm{A} $
	\begin{enumerate}
		\item $ \lambda_{1}\neq \lambda_{2}\Rightarrow \xi_{1}, \xi_{2}$无关
		\item $ \lambda_{1}= \lambda_{2}\Rightarrow \xi_{1}, \xi_{2}$
		\begin{enumerate}
			\item $\xi_{1}, \xi_{2}$无关
			\item $\xi_{1}, \xi_{2}$相关
		\end{enumerate}
	\end{enumerate}
    \item 实对称矩阵$ \bm{A} $
    \begin{enumerate}
    	\item $ \lambda_{1}\neq \lambda_{2}\Rightarrow \xi_{1}\perp \xi_{2}$\quad $\xi_{1}, \xi_{2}$无关
    	\item $ \lambda_{1}= \lambda_{2}\Rightarrow$ 
    	\begin{enumerate}
    		\item $\xi_{1}\perp \xi_{2}$\quad $\xi_{1}, \xi_{2}$无关
    		\item $\xi_{1}$不垂直于$\xi_{2}$\quad $\xi_{1}, \xi_{2}$无关
    	\end{enumerate}
    \end{enumerate}
\end{enumerate}
\chapter{二次型}
\section{基础知识}
\subsection{二次型}
\subsubsection{定义}
$ n $元变量$ x_{1}, x_{2},... ,x_{n} $的二次齐次多项式称为$ n $元二次型, 简称二次型.\par
二次型可以表示为$ \sum_{i=1}^{n}\sum_{j=1}^{n}a_{ij}x_{i}x_{j} $, 由此可以得出二次型的矩阵表达式, 令:
\begin{equation*}
	\bm{A}=\begin{bmatrix}
		a_{11} & a_{12} & \dots & a_{1n} \\
		a_{21} & a_{22} & \dots & a_{2n} \\
		\vdots & \vdots &  & \vdots \\
		a_{n1} & a_{n2} & \dots & a_{nn} 
	\end{bmatrix},
    \bm{x}=\begin{bmatrix}
    	x_1 \\
    	x_2 \\
    	\vdots \\
    	x_n 
    \end{bmatrix}
\end{equation*}\par
则二次型可以表示为:
\begin{equation*}
	f(\bm{x})=\bm{x}^{T}\bm{A}\bm{x}
\end{equation*}













\end{document} 