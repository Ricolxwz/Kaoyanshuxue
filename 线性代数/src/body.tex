\chapter{行列式}
\section{基础知识}
\subsection{行列式}
\subsubsection{定义}
\begin{enumerate}
\item 几何定义\par $ n $阶行列式为$ n $个$ n $维向量组成的$ n $维图形的体积.
\item 逆序数法定义
\begin{equation*}
\begin{vmatrix}
a_{11}	&a_{12}  &\dots  &a_{1n}  \\
a_{21}	&a_{22}  &\dots  &a_{2n}  \\
\vdots	&\vdots  &\ddots  &\vdots  \\
a_{n1}	&a_{n2}  &\dots  &a_{nn}  
\end{vmatrix}  = \sum_{j_{1},j_{2}...j_{n}}(-1)^{r(j_{1}j_{2}...j_{n})}a_{1j_{1}}a_{2j_{2}}...a_{nj_{n}}
\end{equation*}\par
总共有$ n! $个项.
\item 展开定义\par 代数余子式: $ \bm{A}_{ij}=(-1)^{i+j}\bm{M}_{ij} $\par 按第$ i $行展开: $ a_{i1}\bm{A}_{i1}+a_{i2}\bm{A}_{i2}+...+a_{in}\bm{A}_{in} $\par 注意, 行列式的某行(列)元素分别乘另一行(列)的元素的代数余子式后再求和为$ 0 $
\end{enumerate}
\subsubsection{性质}
\begin{enumerate}
\item $ |\bm{A}^{T}|=|\bm{A}| $, 若$ \bm{A} = \bm{A}^{T} $, 则矩阵$ \bm{A} $为对称矩阵, 若$ \bm{A} \times \bm{A}^{T} = \bm{E} $, 则矩阵$ \bm{A} $为正交矩阵
\item 若行列式中某行(列)全部元素为$ 0 $, 行列式为$ 0 $
\item 若行列式中某行(列)元素有公因子$ k $($ k\neq 0 $), $ k $可以提到行列式外面
\item 行列式某行(列)元素均是两个元素只和, 可以拆成两个行列式只和
\item 两行(列)互换, 值取反
\item 两行(列)元素对应成比例, 行列式为$ 0 $
\item 行列式中某行(列)$ k $倍加到另一行(列), 值不变
\end{enumerate}
\subsubsection{重要行列式}
\begin{enumerate}
\item 主对角线行列式(上/下三角形行列式): $ |\bm{A}|=\prod_{i=1}^{n}a_{ii} $
\item 副对角线行列式: $ |\bm{A}|=(-1)^{\frac{n(n-1)}{2}}a_{1n}a_{2,n-1}...a_{n1} $
\item 拉普拉斯展开式
\subitem $ \bm{A} $为$ m $阶矩阵, $ \bm{B} $为$ n $阶矩阵
\subitem 主对角线:
$ \begin{vmatrix}
\bm{A} & \bm{0} \\
\bm{0} & \bm{B}
\end{vmatrix} = \begin{vmatrix}
\bm{A} & \bm{C} \\
\bm{0} & \bm{B}
\end{vmatrix} = \begin{vmatrix}
\bm{A} & \bm{0} \\
\bm{C} & \bm{B}
\end{vmatrix} = |\bm{A}||\bm{B}|$
\subitem 副对角线: 
$ \begin{vmatrix}
\bm{0} & \bm{A} \\
\bm{B} & \bm{0}
\end{vmatrix} = \begin{vmatrix}
\bm{C} & \bm{A} \\
\bm{B} & \bm{0}
\end{vmatrix} = \begin{vmatrix}
\bm{0} & \bm{A} \\
\bm{B} & \bm{C}
\end{vmatrix} = (-1)^{mn}|\bm{A}||\bm{B}|$
\item 范特蒙德行列式
\begin{equation*}
\begin{vmatrix}
1 & 1 & \dots & 1 \\
x_1 & x_2 & \dots & x_n \\
x_1^2 & x_2^2 & \dots & x_n^2 \\
\vdots & \vdots  & \vdots & \vdots \\
x_{1}^{n-1} & x_{2}^{n-1} & \dots & x_{n}^{n-1} 
\end{vmatrix} = \sum_{1\le i<j\le n}(x_{j}-x_{i})
\end{equation*}\par
 注意, 范氏行列式中全$1$行一定在上方.
\end{enumerate}
\section{习题}
\subsection{行列式的计算}
\subsubsection{具体型行列式}
\begin{enumerate}
\item 化基本形法
\begin{enumerate}
\item 直接展开: 适用于含$ 0 $较多的行(列)
\item 爪型: 斜爪消平爪
\item 异爪型
\begin{enumerate}
\item 阶数不高, 直接展开
\item 阶数高, 用递推(尤其适用于一横形行列式)\par
\paragraph{例题} 一横形行列式
$\begin{bmatrix}
1-x & x & 0 \\
-1 & 1-x & x \\
0 & -1 & 1-x
\end{bmatrix}$=(\qquad)
\paragraph{解} 将其按第行展开, 得到一个相似的行列式, 可以得到一个递推公式.
\end{enumerate}
\item 行(列)和相等: 三种方法
\begin{enumerate}
\item 提取公因子: 将其余行全都加到第一行上去, 提取公因子
\item 加边法: 例如矩阵
\begin{equation*}
\begin{bmatrix}
a_{1}-b & a_{2} & \dots & a_{n} \\
a_{1} & a_{2}-b & \dots & a_{n} \\
\vdots & \vdots & \ddots & \vdots \\
a_{1} & a_{2} & \dots & a_{n}-b
\end{bmatrix}=
\begin{bmatrix}
1 & a_{1} & a_{2} & \dots & a_{n} \\
0 & a_{1}-b & a_{2} & \dots & a_{n} \\
0 & a_{1} & a_{2}-b & \dots & a_{n} \\
0 & \vdots & \vdots & \ddots & \vdots \\«
0 & a_{1} & a_{2} & \dots & a_{n}-b
\end{bmatrix}
\end{equation*}加边后矩阵的值不变, 可以将第$ 1 $行的$ -1 $倍加到其他行, 再用其他行的$ (-1/b) $倍加到第一列.
\item 化爪形行列式
\end{enumerate}
\item 消零化基本形: 
\item 拉普拉斯行列式: 一般为``X字形''
\item 范特蒙德行列式: 化为范式行列式, 看第二行写结果
\end{enumerate}
\item 递推法
\item 行列式表示的函数和方程
\end{enumerate}
\subsubsection{抽象型行列式}
\begin{enumerate}
\item 目标行列式和矩阵的相互转换: $|\bm{A}\bm{B}|=|\bm{A}||\bm{B}|$
\paragraph{例题} 设$\bm{\alpha_{1}},\bm{\alpha_{2}},...,\bm{\alpha_{n}}$是$n$维向量, $\bm{A}=[\bm{\alpha_{1}},\bm{\alpha_{2}},...,\bm{\alpha_{n}}],\bm{B}=[\bm{\alpha_{n}},\bm{\alpha_{1}},\bm{\alpha_{2}},...,\bm{\alpha_{n-1}}]$. 若$|\bm{A}|=1$, 则$|\bm{A}-\bm{B}|=(\quad)$\par
\item 与特征方程相结合
\paragraph{例题} 设$\bm{A}$是$ 3 $阶方阵, 满足$ |3 \bm{A}+2 \bm{E}|=0, |\bm{A} - \bm{E}|=0, |3 \bm{E}-2 \bm{A}|=0 $, 则$ |\bm{A}|=(\qquad) $
\paragraph{解} 特征方程$ |\lambda \bm{E}-\bm{A}|=0 $, 可以根据上面的几个等式求出矩阵$ \bm{A} $的特征值, 根据特征值的性质可以知道矩阵的季和矩阵对应行列式的值
\end{enumerate}
\subsection{余子式和代数余子式的线性组合的计算}
根据行列式的展开定义, 有:
\begin{equation*}
a_{i1}A_{i1}+a_{i2}A_{i2}+...+a_{in}A_{in}=
\begin{bmatrix}
 &  & \dots &  & \\
a_{i1} & a_{i2} & \dots & \dots & a_{in} \\
 &  & \dots &  & \\
\end{bmatrix}
\end{equation*}\par
则有:
\begin{equation*}
k_{1}A_{i1}+k_{2}A_{i2}+...+k_{i1}A_{in}=
\begin{bmatrix}
&  & \dots &  & \\
k_{i1} & k_{i2} & \dots & \dots & k_{in} \\
&  & \dots &  & \\
\end{bmatrix}
\end{equation*}\par
\paragraph{例题} 设$\bm{|A|}=
\begin{bmatrix}
2 & -1 & 2 & 3 \\
0 & -1 & -1 & 0 \\
2 & 3 & 4 & 5 \\
1 & 1 & 1 & 1
\end{bmatrix}
$, 则$A_{31}+A_{32}+A_{33}+M_{34}=
\begin{bmatrix}
2 & -1 & 2 & 3 \\
0 & -1 & -1 & 0 \\
1 & 1 & 1 & -1 \\
1 & 1 & 1 & 1
\end{bmatrix}
$
\section{总结}
\subsection{重点}
\begin{enumerate}
\item 一横形行列式的计算
\item 行(列)和相等行列式的计算
\item 余子式和代数余子式的计算
\item 结合特征方程
\end{enumerate}
\chapter{矩阵}
\section{基础知识}
\subsection{矩阵}
\subsubsection{本质}
矩阵的本质是表达系统信息.
\subsubsection{定义}
由$ m\times n $个数排成的$ m $行$ n $列的矩形表格. 当$ m=n $的时候称$ \bm{A} $为$ n $阶方阵.\par 有两个矩阵, 如果$ m,n $相同, 称为同型矩阵.
\subsubsection{运算}
\begin{enumerate}
\item 相等: 同型矩阵且对应元素相等
\item 加法: 同型矩阵对应元素相加
\item 数乘矩阵(重要, 与行列式不同)\par 
\begin{equation*}
k\bm{A} = \bm{A}k = k \begin{bmatrix}
a_{11}	& a_{12}  & \dots & a_{1n} \\
a_{21}	& a_{22} & \dots & a_{2n} \\
\vdots	& \vdots & \ddots & \vdots \\
a_{m1}	& a_{m2} & \dots & a_{mn} 
\end{bmatrix} = 
\begin{bmatrix}
ka_{11}	& ka_{12}  & \dots & ka_{1n} \\
ka_{21}	& ka_{22} & \dots & ka_{2n} \\
\vdots	& \vdots & \ddots & \vdots \\
ka_{m1}	& ka_{m2} & \dots & ka_{mn} 
\end{bmatrix}
\end{equation*}
加法运算和数乘运算统称为矩阵的线性运算, 满足以下运算规律:
\begin{enumerate}
\item 交换律: $ \bm{A}+\bm{B}=\bm{B}+\bm{A} $
\item 结合律: $ (\bm{A}+\bm{B})+\bm{C}=\bm{A}+(\bm{B}+\bm{C}) $
\item 分配律: $ k(\bm{A}+\bm{B})=k\bm{A}+k\bm{B}, (k+l)\bm{A}=k\bm{A}+l\bm{B} $
\item 数和矩阵相乘的结合律: $ k(l\bm{A})=(kl)\bm{A}=l(k\bm{A}) $
\end{enumerate}
\item 乘法: $ \bm{A} $为$ m\times s $矩阵, $ \bm{B} $为$ s\times n $矩阵, 设$ \bm{C}=\bm{A}\bm{B}=(c_{ij})_{m\times n} $
\begin{equation*}
c_{ij}=\sum_{k=1}^{s}a_{ik}b_{kj}=a_{i1}b_{1j}+a_{i2}b_{2j}+...+a_{is}b_{sj}(i=1,2,...,m;j=1,2,...,n)
\end{equation*}
乘法满足下列运算规律:
\begin{enumerate}
\item 结合律: $ (\bm{A}\bm{B})\bm{C}=\bm{A}(\bm{B}\bm{C}) $
\item 分配律: $ \bm{A}(\bm{B}+\bm{C})=\bm{A}\bm{B}+\bm{A}\bm{C} $
\item 数乘与矩阵乘积的结合律: $ (k\bm{A})\bm{B}=\bm{A}(k\bm{B}) $
\end{enumerate}
\item 转置矩阵: 行列互换
\item 向量的内积和正交
\begin{enumerate}
\item 内积: $ \bm{\alpha} = [\alpha_{1},...,\alpha_{n}]^{T}, \bm{\beta} = [\beta_{1},...,\beta_{n}]^{T} $, 内积为
\begin{equation*}
\bm{\alpha}^{T}\bm{\beta}=\sum_{i=1}^{n}a_{i}b_{i}=a_{1}b_{1}+...+a_{n}b_{n}
\end{equation*}
记为$ (\bm{\alpha},\bm{\beta}) $
\item 正交: 内积为$ 0 $
\item 模: 向量的长度, 记作$ ||\bm{\alpha}|| $
\end{enumerate}
\item 标准正交向量组: 所有成员两两正交且模都为$ 1 $, 即:
\begin{equation*}
\begin{aligned}
&\bm{\alpha}_{i}^{T}\bm{\alpha}_{j}=0\ (i\neq j)\\&\bm{\alpha}_{i}^{T}\bm{\alpha}_{j}=1\ (i=j)
\end{aligned}
\end{equation*}
称$ \bm{\alpha_{1}},...,\bm{\alpha_{n}} $为单位正交向量组.
\item 标准正交矩阵: 由标准正交向量组组成的矩阵
\item 施密特正交化(正交规范化)\par
\begin{equation*}
\begin{aligned}
&\bm{\beta_{1}}=\bm{\alpha_{1}}\\&\bm{\beta_{2}}=\bm{\alpha_{2}}-\frac{(\bm{\alpha_{2}},\bm{\beta_{1}})}{(\bm{\beta_{1}},\bm{\beta_{1}})}\bm{\beta_{1}}
\end{aligned}
\end{equation*}
上式得到的是正交向量组, 再进行单位化:\par
\begin{equation*}
\bm{\eta_{1}}=\frac{\bm{\beta_{1}}}{||\bm{\beta_{1}}||}, \bm{\eta_{2}}=\frac{\bm{\beta_{2}}}{||\bm{\beta_{2}}||}
\end{equation*}\par
得到标准正交向量组.
\item 幂: $ \bm{A} $为一个$ n $阶方阵, 则$ \bm{A}^{n}=\bm{A}\bm{A}...\bm{A}\bm{A}(\text{共}n\text{个}\bm{A}) $
\item 方阵乘积的行列式
\begin{equation*}
|\bm{A}\bm{B}|=|\bm{A}||\bm{B}|
\end{equation*}
\end{enumerate}
\subsubsection{重要矩阵}
\begin{enumerate}
\item 零矩阵
\item 单位矩阵
\item 数量矩阵: 数$ k $和单位矩阵的乘积
\item 对角矩阵: 非主对角线元素均为$ 0 $的矩阵
\item 上(下)三角矩阵
\item 对称矩阵: $ \bm{A}^{T}=\bm{A} $
\item 反对称矩阵: $ \bm{A}^{T}=-\bm{A} $
\item 标准正交矩阵: $ \bm{A}^{T}\bm{A}=\bm{E} $, 即行(列)向量的组合是标准正交向量组
\item 分块矩阵\par
分块矩阵的加法和数乘与行列式不同:
\begin{enumerate}
\item 加法
\begin{equation*}
\begin{bmatrix}
\bm{A_{1}} & \bm{A_{2}} \\ \bm{A_{3}} & \bm{A_{4}}
\end{bmatrix} + 
\begin{bmatrix}
\bm{B_{1}} & \bm{B_{2}} \\ \bm{B_{3}} & \bm{B_{4}}
\end{bmatrix} = 
\begin{bmatrix}
\bm{A_{1}}+\bm{B_{1}} & \bm{A_{2}}+\bm{B_{2}} \\ \bm{A_{3}}+\bm{B_{3}} & \bm{A_{4}}+\bm{B_{4}}
\end{bmatrix}
\end{equation*}
\item 数乘
\begin{equation*}
k\begin{bmatrix}
\bm{A} & \bm{B} \\ \bm{C} & \bm{D}
\end{bmatrix} = 
\begin{bmatrix}
k\bm{A} & k\bm{B} \\ k\bm{C} & k\bm{D}
\end{bmatrix}
\end{equation*}
\item 乘法: 与矩阵乘法相同
\end{enumerate}
\end{enumerate}
\subsection{矩阵的逆}
\subsubsection{定义}
若$ \bm{A}\bm{B}=\bm{B}\bm{A}=\bm{E} $, 则矩阵$ \bm{A} $可逆, $ \bm{B} $为$ \bm{A} $的逆矩阵.
\subsubsection{性质}
设$ \bm{A}, \bm{B} $为同阶可逆矩阵
\begin{enumerate}
\item $ (k\bm{A})^{-1}=\frac{1}{k}\bm{A}^{-1}\ (k\neq 0) $
\item $ (\bm{A}\bm{B})^{-1}=\bm{B}^{-1}\bm{A}^{-1}\ (\bm{A}\bm{B}\text{也可逆}) $
\item $ (\bm{A}^{T})^{-1}=(\bm{A}^{-1})^{T}\ (\bm{A}^{T}\text{也可逆}) $
\item $ (\bm{A}+\bm{B})^{-1}\neq \bm{A}^{-1}+\bm{B}^{-1} $
\item $ (\bm{A}^{-1})^{-1}=\bm{A} $
\item $ |\bm{A}^{-1}|=|\bm{A}|^{-1} $ \par
推导: $|\bm{A}^{-1}\bm{A}|=|\bm{A}^{-1}||\bm{A}|=1$
\end{enumerate}
\subsection{伴随矩阵}
\subsubsection{定义}
矩阵$ \bm{A} $的伴随矩阵为:
\begin{equation*}
\bm{A}^{*}=
\begin{bmatrix}
A_{11}	& A_{21} & \dots & A_{n1} \\
A_{12}	& A_{22} & \dots & A_{n2} \\
\vdots	& \vdots & \ddots & \vdots \\
A_{1n}	& A_{2n} & \dots & A_{nn}
\end{bmatrix}
\end{equation*}\par
其中$ \bm{A} $为对应元素的代数余子式.
\subsubsection{性质}
\begin{enumerate}
\item $ \bm{A}\bm{A}^{*}=\bm{A}^{*}\bm{A}=|\bm{A}|\bm{E} $
\item $ |\bm{A}^{*}|=|\bm{A}|^{n-1} $
\item $ (\bm{A}^{*})^{*}=|\bm{A}|^{n-2}\bm{A} $
\item $ (\bm{A})^{-1}=\frac{\bm{A}^{*}}{|\bm{A}|} $
\end{enumerate}
\subsection{初等矩阵}
\subsubsection{定义}
单位矩阵经过一次初等变换后得到的矩阵称为初等矩阵, 有三种:
\begin{enumerate}
\item 倍乘初等矩阵
\begin{equation*}
\bm{E}_{2}(k)=
\begin{bmatrix}
1 & 0 & 0 \\
0 & k & 0 \\
0 & 0 & 1
\end{bmatrix}
\end{equation*}
\item 互换初等矩阵
\begin{equation*}
\bm{E}_{12}=
\begin{bmatrix}
0 & 1 & 0 \\
1 & 0 & 0 \\
0 & 0 & 1
\end{bmatrix}
\end{equation*}
\item 倍加初等矩阵
\begin{equation*}
\bm{E}_{31}(k)=
\begin{bmatrix}
1 & 0 & 0 \\
0 & 1 & 0 \\
k & 0 & 1
\end{bmatrix}
\end{equation*}\par 注意是第一行的$ k $倍加到第三行或者是第三列的$ k $倍加到第一列(别搞错顺序).
\end{enumerate}
\subsubsection{性质}
\begin{enumerate}
\item $ [\bm{E}_{ij}]^{T}=\bm{E}_{ij}, [\bm{E}_{i}(k)]^{T}=\bm{E}_{i}(k), [\bm{E}_{ij}(k)]^{T}=\bm{E}_{ji}(k) $
\item $ [\bm{E}_{ij}]^{-1}=\bm{E}_{ij}, [\bm{E}_{i}(k)]^{-1}=\bm{E}_{i}(\frac{1}{k}), [\bm{E}_{ij}(k)]^{-1}=\bm{E}_{ij}(-k) $
\item 左行右列定理
\item 若$ \bm{A} $为可逆矩阵, 则可以表示为有限个可逆矩阵的乘积
\end{enumerate}
\subsection{等价矩阵}
若$ \bm{A}, \bm{B} $均为$ m\times n $矩阵, 且$ r(\bm{A})=r(\bm{B}) $, 则$ \bm{A}, \bm{B} $为等价矩阵, 记作$ \bm{A}\cong\bm{B} $.
\subsection{矩阵的秩}
\subsubsection{定义}
设$ \bm{A} $为$ m\times n $矩阵, $ \bm{A} $中最高阶非零子式的阶数为矩阵$ \bm{A} $的秩.
如果$ \bm{A} $为$ n\times n $矩阵, 则$ r(\bm{A})=n\ (\text{满秩})\Leftrightarrow |\bm{A}|\neq 0\Leftrightarrow \bm{A}\text{可逆} $.
\subsubsection{初等变换不改变矩阵的秩}
$ r(\bm{A})=r(\bm{PA})=r(\bm{AQ})=r(\bm{PAQ}) $
\subsubsection{重要式子}
设$ \bm{A} $为$ m\times n $矩阵, $ \bm{B} $为满足有关矩阵运算要求的矩阵, 则
\begin{enumerate}
\item $ 0\le r(\bm{A})\le \text{min}\{m,n\} $
\item 数乘: $ r(k\bm{A})=r(\bm{A}) $
\item $ r(\bm{A}\bm{B})\le \text{min}\{r(\bm{A}),r(\bm{B})\} $
\item $ r(\bm{A}+\bm{B})\le r(\bm{A})+r(\bm{B}) $
\item $ r(\bm{A}^{*})=\left\{\begin{aligned}
& n\quad r(\bm{A})=n \\
& 1\quad r(\bm{A})=n-1 \\
& 0\quad r(\bm{A})<n-1
\end{aligned}\right.$ 其中$ \bm{A} $为$ n $阶方阵
\end{enumerate}
\subsection{常见运算汇总}
\begin{enumerate}
\item 
\subitem $ |k\bm{A}|=k^{n}\bm{A} $
\subitem $ (k\bm{A})^{T}=k\bm{A}^{T} $
\subitem $ (k\bm{A})^{-1}=\frac{1}{k}\bm{A}^{-1} $
\subitem $ (k\bm{A})^{*}=k^{n-1}\bm{A}^{*} $
\item 
\subitem $ |\bm{A}+\bm{B}|\neq |\bm{A}|+|\bm{B}| $
\subitem $ (\bm{A}+\bm{B})^{T}=\bm{A}^{T}+\bm{B}^{T} $
\subitem $ (\bm{A}+\bm{B})^{-1}\neq \bm{A}^{-1}+\bm{B}^{-1} $
\subitem $ (\bm{A}+\bm{B})^{*}\neq \bm{A}^{*}+\bm{B}^{*} $
\item 
\subitem $ |\bm{A}\bm{B}|=|\bm{A}||\bm{B}| $
\subitem $ (\bm{A}\bm{B})^{T}=\bm{B}^{T}\bm{A}^{T} $
\subitem $ (\bm{A}\bm{B})^{-1}=\bm{B}^{-1}\bm{A}^{-1} $
\subitem $ (\bm{A}\bm{B})^{*}=\bm{B}^{*}\bm{A}^{*} $
\item 
\subitem $ (\bm{A}^{-1})^{*}=(\bm{A}^{*})^{-1} $
\subitem $ (\bm{A}^{-1})^{T}=(\bm{A}^{T})^{-1} $
\subitem $ (\bm{A}^{T})^{*}=(\bm{A}^{*})^{T} $
\end{enumerate}
\section{习题}
\chapter{向量组}
\section{基础知识}
\subsection{向量}
\subsubsection{定义}
$ n $个数构成的一个有序数组$ [a_{1},a_{2},...,a_{n}] $称为一个$ n $维向量, 记为$ \bm{\alpha} = [a_{1},a_{2},...,a_{n}] $, 并称$ \bm{\alpha} $为$ n $维行向量, $ \bm{\alpha}^{T} $为$ n $维列向量. 其中$ a_{i} $称为向量$ \bm{\alpha} $或者$ \bm{\alpha}^{T} $的第$ i $个分量.
\subsection{线性组合和线性相关}
\subsubsection{定义}
\begin{enumerate}
\item 线性组合: 设有$ m $个$ n $维向量$ \bm{\alpha_{1}},\bm{\alpha_{2}},...,\bm{\alpha_{m}} $和$ m $个数$ k_{1},k_{2},...,k_{m} $. 则向量$ k_{1}\bm{\alpha_{1}}+k_{2}\bm{\alpha_{2}}+...+k_{m}\bm{\alpha_{m}} $为向量组$ \bm{\alpha_{1}},\bm{\alpha_{2}},...,\bm{\alpha_{m}} $的线性组合
\item 线性表出: 若向量$ \bm{\beta} $能表示成向量组$ \bm{\alpha_{1}},\bm{\alpha_{2}},...,\bm{\alpha_{m}} $的线性组合, 即$ \bm{\beta}=k_{1}\bm{\alpha_{1}}+k_{2}\bm{\alpha_{2}}+...+k_{m}\bm{\alpha_{m}} $, 则称$ \bm{\beta} $能够被向量组线性表出
\item 线性相关: 存在一组不全为$ 0 $的数$ k_{1},k_{2},...,k_{m} $, 使得下式成立:
\begin{equation*}
k_{1}\bm{\alpha_{1}}+k_{2}\bm{\alpha_{2}}+...+k_{m}\bm{\alpha_{m}}=\bm{0}
\end{equation*}\par
上式可以进一步写为$ x_{1}\bm{\alpha_{1}}+x_{2}\bm{\alpha_{2}}+...+x_{m}\bm{\alpha_{m}}=\bm{0} $, 这个式子有四种形式:
\begin{equation*}
\bm{A}\bm{x}=\bm{0}
\end{equation*}
或者矩阵形式:
\begin{equation*}
\bm{A}\bm{x}=\begin{bmatrix}
a_{11} & a_{12} & \dots & a_{1m} \\
a_{21} & a_{22} & \dots & a_{2m} \\
\vdots & \vdots & \ddots & \vdots \\
a_{n1} & a_{n2} & \dots & a_{nm}
\end{bmatrix}
\begin{bmatrix}
x_{1} \\
x_{2} \\
\vdots \\
x_{m}
\end{bmatrix}=
\begin{bmatrix}
0 \\
0 \\
\vdots \\
0
\end{bmatrix}
\end{equation*}
或者向量形式:
\begin{equation*}
\bm{A}\bm{x}=x_{1}\begin{bmatrix}
a_{11} \\
a_{21} \\
\vdots \\
a_{n1}
\end{bmatrix}+
x_{2}\begin{bmatrix}
a_{12} \\
a_{22} \\
\vdots \\
a_{n2}
\end{bmatrix}+...+
x_{m}\begin{bmatrix}
a_{1m} \\
a_{2m} \\
\vdots \\
a_{nm}
\end{bmatrix}=
\begin{bmatrix}
0 \\
0 \\
\vdots \\
0
\end{bmatrix}
\end{equation*}
或者线性方程组形式:
\begin{equation*}
\left\{\begin{aligned}
& x_{1}a_{11}+x_{2}a_{12}+...+x_{m}a_{1m}=0 \\
& x_{1}a_{21}+x_{2}a_{22}+...+x_{m}a_{2m}=0 \\
& \dots \\
& x_{1}a_{n1}+x_{2}a_{n2}+...+x_{m}a_{nm}=0
\end{aligned}
\right.
\end{equation*}
\item 线性无关: 只有当$ k_{1},k_{2},...,k_{m} $全为$ 0 $的时候, 才能使上式成立
\end{enumerate}
\subsubsection{判别相关性定理}
\begin{enumerate}
\item 相关充要条件: 向量组中至少有一个向量能被其余的$ n-1 $的向量线性表出
\item 相关充要条件: 方程$ \bm{A}\bm{x}=\bm{0} $有非$ \bm{0} $解
\item 若向量组$ \bm{\alpha_{1}},\bm{\alpha_{2}},...,\bm{\alpha_{m}} $线性无关, 而向量组$ \bm{\beta}, \bm{\alpha_{1}},\bm{\alpha_{2}},...,\bm{\alpha_{m}} $线性相关, 则$ \bm{\beta} $可由向量组$ \bm{\alpha_{1}},\bm{\alpha_{2}},...,\bm{\alpha_{m}} $线性表出, 且表示方法唯一\label{ref:判别相关性}
\item 如果向量$ \bm{\beta} $能够由向量组$ \bm{\alpha_{1}},\bm{\alpha_{2}},...,\bm{\alpha_{m}} $线性表出, 则$ r([\bm{\alpha_{1}},\bm{\alpha_{2}},...,\bm{\alpha_{m}}])=r([\bm{\beta},\bm{\alpha_{1}},\bm{\alpha_{2}},...,\bm{\alpha_{m}}]) $
\item 以少表多, 多的相关: 如果向量组$ \bm{\beta_{1}},\bm{\beta_{2}},...,\bm{\beta_{t}} $能够由向量组$ \bm{\alpha_{1}},\bm{\alpha_{2}},...,\bm{\alpha_{s}} $线性表示, 且$ t>s $, 则$ \bm{\beta_{1}},\bm{\beta_{2}},...,\bm{\beta_{t}} $线性相关
\item 向量组的部分与整体:
\subitem 如果向量组$ \bm{\alpha_{1}},\bm{\alpha_{2}},...,\bm{\alpha_{m}} $中有一部分向量线性相关, 则整体也线性相关;
\subitem 如果向量组$ \bm{\alpha_{1}},\bm{\alpha_{2}},...,\bm{\alpha_{m}} $线性无关, 则其任一部分线性无关
\item 向量的部分与整体:
\subitem 如果向量组$ \bm{\alpha_{1}},\bm{\alpha_{2}},...,\bm{\alpha_{m}} $线性无关, 则将所有向量扩展到$ s $维得到的向量组$ \bm{\alpha_{1}}^{*},\bm{\alpha_{2}}^{*},...,\bm{\alpha_{m}}^{*} $也是线性无关的;
\subitem 如果向量组$ \bm{\alpha_{1}},\bm{\alpha_{2}},...,\bm{\alpha_{m}} $线性相关, 则将所有向量缩减到$ k $维得到的向量组$ \bm{\alpha_{1}}^{*},\bm{\alpha_{2}}^{*},...,\bm{\alpha_{m}}^{*} $也是线性相关的
\end{enumerate}
\subsection{极大线性无关组和等价向量组}
\subsubsection{定义}
在向量组$ \bm{\alpha_{1}},\bm{\alpha_{2}},...,\bm{\alpha_{m}} $中, 存在向量组$ \bm{\alpha_{i_{1}}},\bm{\alpha_{i_{2}}},...,\bm{\alpha_{i_{s}}} $, 满足以下条件:
\begin{enumerate}
\item $ \bm{\alpha_{i_{1}}},\bm{\alpha_{i_{2}}},...,\bm{\alpha_{i_{s}}} $线性无关
\item 向量组$ \bm{\alpha_{1}},\bm{\alpha_{2}},...,\bm{\alpha_{m}} $中的任一向量能够由向量组$ \bm{\alpha_{i_{1}}},\bm{\alpha_{i_{2}}},...,\bm{\alpha_{i_{s}}} $线性表示
\end{enumerate}\par
则称向量组$ \bm{\alpha_{i_{1}}},\bm{\alpha_{i_{2}}},...,\bm{\alpha_{i_{s}}} $为原向量组的极大线性无关组.
\subsection{等价向量组}
\subsubsection{定义}
若有两个向量组$ (1)\ \bm{\alpha_{1}},\bm{\alpha_{2}},...,\bm{\alpha_{s}} $和$ (2)\ \bm{\beta_{1}},\bm{\beta_{2}},...,\bm{\beta_{t}} $, 这两个向量组中的任一元素都可以由另一向量组线性表出, 则称这两个向量组为等价向量组.
\subsubsection{性质}
\begin{enumerate}
\item 反身性: $ (1)\simeq (1) $
\item 对称性: $ (1)\simeq (2) \Leftrightarrow (2)\simeq (1) $
\item 传递性: $ (1)\simeq (2), (2)\simeq (3) \Rightarrow (1)\simeq (3) $
\item 向量组和它的极大线性无关组是等价向量组
\item 等价向量组有相等的秩
\end{enumerate}
\subsection{向量组的秩}
\subsubsection{定义}
向量组的秩是极大线性无关组成员的个数, 是线性无关向量的个数, 是向量空间的维数, 是最简化的向量数.
\subsubsection{性质}
\begin{enumerate}
\item 三秩相等: $ r(\bm{A}) $矩阵的秩=$ \bm{A} $的行秩=$ \bm{A} $的列秩
\item 若$ \bm{A}\xrightarrow{\text{初等行变换}}\bm{B}$, 则
\begin{enumerate}
\item $ \bm{A} $的行向量组和$ \bm{B} $的行向量组是等价向量组
\item $ \bm{A} $和$ \bm{B} $的任何相应部分列向量具有相同的线性相关性
\end{enumerate}
\item 设向量组$ (1)\ \bm{\alpha_{1}},\bm{\alpha_{2}},...,\bm{\alpha_{s}} $和$ (2)\ \bm{\beta_{1}},\bm{\beta_{2}},...,\bm{\beta_{t}} $, 若$ \bm{\beta}_{i} $均可由$ \bm{\alpha_{1}},\bm{\alpha_{2}},...,\bm{\alpha_{s}} $线性表出, 则$ r(\bm{\beta_{1}},\bm{\beta_{2}},...,\bm{\beta_{t}})\le r(\bm{\alpha_{1}},\bm{\alpha_{2}},...,\bm{\alpha_{s}}) $ \par
可以这么理解: 秩其实就是一种多样性, 多样的数据的集合肯定能够表示单一的数据的集合, 即如果秩越大, 则这些数据的多样性就越大. 所以$\bm{\alpha_{1}},\bm{\alpha_{2}},...,\bm{\alpha_{s}}$的秩一定大.
\end{enumerate}
\section{习题}
\chapter{线性方程组}
\section{基础知识}
\subsection{齐次线性方程组}
设有一齐次线性方程组:
\begin{equation*}
\left\{\begin{aligned}
& x_{1}a_{11}+x_{2}a_{12}+...+x_{m}a_{1m}=0 \\
& x_{1}a_{21}+x_{2}a_{22}+...+x_{m}a_{2m}=0 \\
& \dots \\
& x_{1}a_{n1}+x_{2}a_{n2}+...+x_{m}a_{nm}=0
\end{aligned}
\right.
\end{equation*}\par
其矩阵形式为:
\begin{equation*}
\bm{A}\bm{x}=\begin{bmatrix}
a_{11} & a_{12} & \dots & a_{1m} \\
a_{21} & a_{22} & \dots & a_{2m} \\
\vdots & \vdots & \ddots & \vdots \\
a_{n1} & a_{n2} & \dots & a_{nm}
\end{bmatrix}
\begin{bmatrix}
x_{1} \\
x_{2} \\
\vdots \\
x_{m}
\end{bmatrix}=
\begin{bmatrix}
0 \\
0 \\
\vdots \\
0
\end{bmatrix}
\end{equation*}
\subsubsection{有解的条件}
由上矩阵可以得到, 未知数的个数为$ m $, 方程的个数为$ n $.
\begin{enumerate}
\item 若$ m>n $, 则必有非零解
\item 若$ m=n $, 用秩判断:
\begin{enumerate}
\item 若$ r(\bm{A})=m $(向量组线性无关, $ |\bm{A}|\neq 0 $), 则仅有零解 \par
说明: 由于行秩=列秩, 所以列秩为$ m $, 说明独立方程组个数为$ m $.
\item 若$ r(\bm{A})=r<m $(向量组线性相关, $ |\bm{A}| = 0 $), 则必有非零解, 且有$ m-r $个线性无关解 \par
说明: 由于行秩=列秩, 所以列秩为$ r $, 说明独立方程组个数为$ r $.
\end{enumerate}
\item 若$ m<n $,
\end{enumerate}
\subsubsection{解的性质}
若$ \bm{A}\bm{\xi_{1}}=\bm{0} $, $ \bm{A}\bm{\xi_{2}}=\bm{0} $, 则$ k_{1}\bm{\xi_{1}}+k_{2}\bm{\xi_{2}}=0 $, 其中$ k_{1},k_{2} $为任意常数.
\subsubsection{基础解系和解的结构}
\begin{enumerate}
\item 基础解系\par
设$ \bm{\xi_{1}},\bm{\xi_{2}},...,\bm{\xi_{m-r}} $满足:
\begin{enumerate}
\item 是方程组$ \bm{A}\bm{x}=\bm{0} $的解
\item 线性无关
\item 方程组$ \bm{A}\bm{x}=\bm{0} $的任一解均可由$ \bm{\xi_{1}},\bm{\xi_{2}},...,\bm{\xi_{m-r}} $线性表出, 则称$ \bm{\xi_{1}},\bm{\xi_{2}},...,\bm{\xi_{m-r}} $ 是方程组$ \bm{A}\bm{x}=\bm{0} $的基础解系
\end{enumerate}
\item 通解\par
设$ \bm{\xi_{1}},\bm{\xi_{2}},...,\bm{\xi_{m-r}} $是方程$ \bm{A}\bm{x}=\bm{0} $的基础解系, 则$ k_{1}\bm{\xi_{1}}+k_{2}\bm{\xi_{2}}+...+k_{m-r}\bm{\xi_{m-r}} $是其通解.
\end{enumerate}
\subsubsection{求解方法}
\begin{enumerate}
\item $ \bm{A}\xrightarrow{\text{初等行变换}}\bm{B} $, 其中$ \bm{B} $为行阶梯形矩阵, $ r(\bm{A})=r $\par
高斯消元法: \par
\ding{172} 保证最靠左的非全$ 0 $列的最上方为非$ 0 $元素, 如果不是, 通过``互换''初等行变换使最靠左非全$ 0 $列的最上方为非$ 0 $元素\par
\ding{173} 通过``倍加''初等行变换使这个非$ 0 $元素所在列的下方元素全为$ 0 $\par
\ding{174} 遮住矩阵的最上面一行不看, 将其余行看作一个新矩阵, 重复\ding{172}\ding{173}, 直至矩阵化为阶梯形\par
高斯-若当消元法:\par
\ding{172} 由高斯消元法得到阶梯形矩阵\par
\ding{173} 对于每一个非全$ 0 $行, 通过``倍乘''初等行变换使得这一行的非$ 0 $首位为$ 1 $\par
\ding{174} 对于每一个非全$ 0 $行, 通过``倍加''初等行变换使得这一行的非$ 0 $首项所在列的上方元素全为$ 0 $, 直至得到简化行阶梯型矩阵
\item 按列找出一个秩为$ r $的子矩阵, 剩余列位置对应的未知数设为自由变量
\item 算出共有$ m-r $个线性无关解, 求出$ \bm{\xi_{1}},\bm{\xi_{2}},...,\bm{\xi_{m-r}} $, 写出通解
\end{enumerate}
\subsection{非齐次线性方程组}
设有一非齐次线性方程组:
\begin{equation*}
\left\{\begin{aligned}
& x_{1}a_{11}+x_{2}a_{12}+...+x_{m}a_{1m}=b_{1} \\
& x_{1}a_{21}+x_{2}a_{22}+...+x_{m}a_{2m}=b_{2} \\
& \dots \\
& x_{1}a_{n1}+x_{2}a_{n2}+...+x_{m}a_{nm}=b_{n}
\end{aligned}
\right.
\end{equation*}\par
其矩阵形式为:
\begin{equation*}
\bm{A}\bm{x}=\begin{bmatrix}
a_{11} & a_{12} & \dots & a_{1m} \\
a_{21} & a_{22} & \dots & a_{2m} \\
\vdots & \vdots & \ddots & \vdots \\
a_{n1} & a_{n2} & \dots & a_{nm}
\end{bmatrix}
\begin{bmatrix}
x_{1} \\
x_{2} \\
\vdots \\
x_{m}
\end{bmatrix}=
\begin{bmatrix}
b_{1} \\
b_{2} \\
\vdots \\
b_{n}
\end{bmatrix}
\end{equation*}\par
特殊的有矩阵$ \bm{A} $的增广矩阵:
\begin{equation*}
[\bm{A},\bm{b}]=\begin{bmatrix}
a_{11} & a_{12} & \dots & a_{1m} & b_{1} \\
a_{21} & a_{22} & \dots & a_{2m} & b_{2} \\
\vdots & \vdots & \ddots & \vdots & \vdots \\
a_{n1} & a_{n2} & \dots & a_{nm} & b_{n}
\end{bmatrix}
\end{equation*}
\subsubsection{有解的条件}
\begin{enumerate}
\item 若$ r(\bm{A})\neq r([\bm{A},\bm{b}]) $($ \bm{b} $不能由$ \bm{\alpha_{1}},\bm{\alpha_{2}},...,\bm{\alpha_{m}} $线性表出), 方程组无解 \par
实际上, $ r([\bm{A},\bm{b}]) = r(\bm{A}) + 1 $.
\item 若$ r(\bm{A})=r([\bm{A},\bm{b}])=m $($ \bm{\alpha_{1}},\bm{\alpha_{2}},...,\bm{\alpha_{m}} $线性无关, $ \bm{\alpha_{1}},\bm{\alpha_{2}},...,\bm{\alpha_{m}},\bm{b} $线性相关), 方程组有唯一解\ref{ref:判别相关性}
\item 若$ r(\bm{A})=r([\bm{A},\bm{b}])=r<m $, 方程组有无穷多解
\end{enumerate}
\subsubsection{解的性质}
设$ \bm{\eta_{1}},\bm{\eta_{2}},\bm{\eta} $是非齐次线性方程组$ \bm{A}\bm{x}=\bm{b} $的解, $ \bm{\xi} $是对应齐次线性方程组$ \bm{A}\bm{x}=\bm{0} $的解, 则:
\begin{enumerate}
\item $ \bm{\eta_{1}}-\bm{\eta_{2}} $是$ \bm{A}\bm{x}=\bm{0} $的解
\item $ k\bm{\xi}+\bm{\eta} $是$ \bm{A}\bm{x}=\bm{b} $的解
\end{enumerate}
\subsubsection{求解方法}
\begin{enumerate}
\item 写出$ \bm{A}\bm{x}=\bm{b} $的导出方程组$ \bm{A}\bm{x}=\bm{0} $, 并求出其通解$ k_{1}\bm{\xi_{1}}+k_{2}\bm{\xi_{2}}+...+k_{m-r}\bm{\xi_{m-r}} $
\item 求出$ \bm{A}\bm{x}=\bm{b} $的一个特解$ \bm{\eta} $
\item $ \bm{A}\bm{x}=\bm{b} $的通解为$ k_{1}\bm{\xi_{1}}+k_{2}\bm{\xi_{2}}+...+k_{m-r}\bm{\xi_{m-r}}+\bm{\eta} $
\end{enumerate}
\chapter{特征值和特征向量}
\section{基础知识}
\subsection{特征值和特征向量}
\subsubsection{定义}
设$ \bm{A} $为$ n $阶矩阵, $ \lambda $是一个数, 若存在一个非零的$ n $维向量$ \bm{\xi} $, 使得$ \bm{A}\bm{\xi}=\lambda \bm{\xi} $, 则称$ \bm{\xi} $为$ \bm{A} $的特征向量, $ \lambda $为$ \bm{A} $的特征值. 
\par 上式可以化简成$ \left| \lambda \bm{E}-\bm{A}\right|=\bm{0} $, $ \left| \lambda \bm{E}-\bm{A}\right| $被称为特征多项式, $ \lambda \bm{E}-\bm{A} $称为特征矩阵. \par
推导: 由于$ (\lambda \bm{E} - \bm{A}) \bm{\xi} = \bm{0} $, 且$ \bm{\xi} \neq \bm{0} $, 说明方程$ (\lambda \bm{E} - \bm{A}) \bm{\xi} = \bm{0} $有非零解, 故$ r(\lambda \bm{E} - \bm{A}) < m $, 即$ |\lambda \bm{E} - \bm{A}| = 0 $.
\subsubsection{性质}
\begin{enumerate}
\item 特征值的性质
\begin{enumerate}
\item $ \sum_{i=1}^{n}\lambda_{i}=\sum_{i=1}^{n}a_{ii}=tr(\bm{A}) $
\item $ \prod_{i=1}^{n}\lambda_{i}=\left| \bm{A}\right| $
\end{enumerate}
\item 特征向量的性质
\begin{enumerate}
\item $ k $重特征值$ \Lambda $至多只有$ k $个线性无关的向量
\item 若$ \bm{\xi_1} $, $ \bm{\xi_2} $是$ \bm{A} $的属于不同特征值的特征的特征向量, 则$ \bm{\xi_1}, \bm{\xi_2} $线性无关
\item 若$ \bm{\xi_1}, \bm{\xi_2} $是$ \bm{A} $的属于同一特征值$ \lambda $的特征向量, 则$ k_1\bm{\xi_1} + k_1\bm{\xi_2} $仍然是$ \bm{A} $的属于特征值$ \lambda $的特征向量
\end{enumerate}
\end{enumerate}
\subsection{矩阵的相似}
\subsubsection{定义}
设$ \bm{A} $和$ \bm{B} $为两个$ n $阶方阵, 若存在$ n $阶可逆矩阵$ \bm{P} $, 使得$ \bm{P}^{-1}\bm{A}\bm{P}=\bm{B} $成立, 则称$ \bm{A} $相似于$ \bm{B} $, 记成$ \bm{A}\sim \bm{B} $.
\subsubsection{性质}
\begin{enumerate}
\item \begin{itemize}
\item 反身性: $ \bm{A}\sim \bm{A} $
\item 对称性: $ \bm{A}\sim \bm{B}\Rightarrow \bm{B}\sim \bm{A} $
\item 传递性: $ \bm{A}\sim \bm{B}, \bm{B}\sim \bm{C}\Rightarrow \bm{A}\sim \bm{C} $
\end{itemize}
\item 若$ \bm{A}\sim \bm{B} $, 则有\begin{itemize}
\item $ r(\bm{A})=r(\bm{B}) $
\item $ \left|\bm{A}\right|=\left|\bm{B}\right| $
\item $ \bm{A}, \bm{B} $具有相同的特征值
\item $ \bm{A}, \bm{B} $特征多项式的值相同
\end{itemize}
\item 若$ \bm{A}\sim \bm{B} $, 则有\begin{itemize}
\item $ f(\bm{A})\sim f(\bm{B}) $
\item $ \bm{A}^{T}\sim \bm{B}^{T} $
\item $ \bm{A} $可逆, $ \bm{A}^{*}\sim \bm{B}^{*} $
\item $ \bm{A} $可逆, $ \bm{A}^{-1}\sim \bm{B}^{-1} $
\end{itemize}
\end{enumerate}
\subsection{矩阵的相似对角化}
\subsubsection{定义}
设$ n $阶矩阵$ \bm{A} $, 存在$ n $阶可逆矩阵$ \bm{P} $, 使得$ \bm{P}^{-1}\bm{A}\bm{P}=\bm{\Lambda} $, 则$ \bm{A}\sim \bm{\Lambda} $, $ \bm{\Lambda} $是$ \bm{A} $的相似标准形.\par 
\[ \bm{P}=\left[\bm{\xi_1}, \bm{\xi_2},... \bm{\xi_n}\right], \bm{\Lambda} =\begin{bmatrix}
\lambda_1 &  &  &  \\
& \lambda_2 &  &  \\
&  & \ddots &  \\
&  &  & \lambda_n 
\end{bmatrix}
\]
\subsubsection{条件}
\begin{enumerate}
\item $ n $阶矩阵$ \bm{A} $可以相似对角化$ \Leftrightarrow $$ \bm{A} $有$ n $个线性无关的特征向量($ \left| \bm{P}\right| = 0$)
\item $ n $阶矩阵$ \bm{A} $可以相似对角化$ \Leftrightarrow $$ \bm{A} $对应于每个$ k_i $重特征值都有$ k_i $个线性无关的特征向量($ n $重特征值对应的解空间是$ n $维)
\item $ n $阶矩阵$ \bm{A} $有$ n $个不同特征值$ \Rightarrow $$ \bm{A} $可以相似对角化(由特征向量的性质3可以推出)
\item $ n $阶矩阵$ \bm{A} $为实对称矩阵$ \Rightarrow $$ \bm{A} $可以相似对角化
\end{enumerate}\par
上述总共两个充要条件, 两个充分条件.
\subsection{实对称矩阵}
\subsubsection{定义}
若$ \bm{A}^T = \bm{A} $, 则$ \bm{A} $为是对称矩阵, 如果在此基础上$ \bm{A} $的元素都是实数, 则$ \bm{A} $是实对称矩阵.
\subsubsection{性质}
\begin{enumerate}\label{ref:实对称矩阵}
\item 实对称矩阵$ \bm{A} $的属于不同特征值的特征向量相互正交
\item 实对称矩阵$ \bm{A} $必相似于对角矩阵, 必有可逆矩阵$ \bm{P}=\left[ \bm{\xi_{1}}, \bm{\xi_{2}},... ,\bm{\xi_{n}}\right] $, 使得$ \bm{P}^{-1}\bm{A}\bm{P}=\bm{\Lambda} $. 且存在正交矩阵$ \bm{Q} $, 使得$ \bm{Q}^{-1}\bm{A}\bm{Q}=\bm{Q}^{T}\bm{A}\bm{Q}=\bm{\Lambda}$,  
\end{enumerate}
\section{习题}
\subsection{特征值和特征向量}
\subsubsection{求具体型矩阵的特征值和特征向量}
\begin{enumerate}
\item 用特征方程$ \left|\lambda \bm{E}-\bm{A}\right|=0 $求出$ \lambda $, 可以使用试根法对$ \lambda $的高次方程进行求解
\item 用求得的$ \lambda $解齐次线性方程组$ (\lambda \bm{E}-\bm{A})\bm{\xi}=\bm{0} $, 求出特征向量
\end{enumerate}
\subsubsection{求解抽象型矩阵的特征值和特征向量}
\begin{table}[H]
\centering
\begin{tabular}{|c|ccccccc|}
\hline
\textbf{矩阵} & $ \bm{A} $ & $ k\bm{A} $ & $ \bm{A}^{k} $ & $ f(\bm{A}) $ & $ \bm{A}^{-1} $ & $ \bm{A}^{*} $ & $ \bm{P}^{-1}\bm{A}\bm{P} $ \\ \hline
\textbf{特征值} & $ \lambda $ & $ k\lambda $ & $ \lambda^{k} $ & $ f(\lambda) $ & $ \lambda^{-1} $ & $ \frac{\left|\bm{A}\right|}{\lambda} $ & $ \lambda $\\ \hline
\textbf{特征向量} & $ \bm{\xi} $ & $ \bm{\xi} $ & $ \bm{\xi} $ & $ \bm{\xi} $ & $ \bm{\xi} $ & $ \bm{\xi} $ & $ \bm{P}^{-1}\bm{\xi} $\\ 
\hline
\end{tabular}
\end{table}\par
$ f(x) $为多项式, 若矩阵$ \bm{A} $满足$ f(\bm{A})=\bm{0}\Rightarrow f(\lambda)=0 $.
\subsection{实对称矩阵}
\subsubsection{求正交矩阵Q}
\begin{enumerate}
\item 求$ \bm{A} $的$ \lambda $与$ \bm{\xi} $
\item $ \bm{\xi_1}, \bm{\xi_{2}},... ,\bm{\xi_{n}} $施密特正交化, 单位化至$ \bm{\eta_{1}}, \bm{\eta_{2}},... ,\bm{\eta_{n}} $
\item 令$ Q=(\bm{\eta_1}, \bm{\eta_2},... ,\bm{\eta_{n}}) $
\end{enumerate}
\par 不同的特征值$ \lambda_{i} $对应的特征矩阵$ \bm{\xi_{i}} $之间是正交的.
\par 施密特正交化:$ \beta_{1}=\alpha_{1}, 
\beta_{2}=\alpha_{2}-\frac{(\alpha_{2}, \beta_{1})}{(\beta_{1}, \beta_{1})}\beta_{1} $.
\par 单位化: $ \eta_{1}=\frac{\beta_{1}}{||\beta_{1}||} $.
\paragraph{总结} 
\begin{enumerate}
\item 普通矩阵$ \bm{A} $
\begin{enumerate}
\item $ \lambda_{1}\neq \lambda_{2}\Rightarrow \xi_{1}, \xi_{2}$无关
\item $ \lambda_{1}= \lambda_{2}\Rightarrow \xi_{1}, \xi_{2}$
\begin{enumerate}
\item $\xi_{1}, \xi_{2}$无关
\item $\xi_{1}, \xi_{2}$相关
\end{enumerate}
\end{enumerate}
\item 实对称矩阵$ \bm{A} $
\begin{enumerate}
\item $ \lambda_{1}\neq \lambda_{2}\Rightarrow \xi_{1}\perp \xi_{2}$\quad $\xi_{1}, \xi_{2}$无关
\item $ \lambda_{1}= \lambda_{2}\Rightarrow$ 
\begin{enumerate}
\item $\xi_{1}\perp \xi_{2}$\quad $\xi_{1}, \xi_{2}$无关
\item $\xi_{1}$不垂直于$\xi_{2}$\quad $\xi_{1}, \xi_{2}$无关
\end{enumerate}
\end{enumerate}
\end{enumerate}
\chapter{二次型}
\section{基础知识}
\subsection{二次型}
\subsubsection{定义}
$ n $元变量$ x_{1}, x_{2},... ,x_{n} $的二次齐次多项式称为$ n $元二次型, 简称二次型.\par
二次型可以表示为$ \sum_{i=1}^{n}\sum_{j=1}^{n}a_{ij}x_{i}x_{j} $, 由此可以得出二次型的矩阵表达式, 令:
\begin{equation*}
\bm{A}=\begin{bmatrix}
a_{11} & a_{12} & \dots & a_{1n} \\
a_{21} & a_{22} & \dots & a_{2n} \\
\vdots & \vdots &  & \vdots \\
a_{n1} & a_{n2} & \dots & a_{nn} 
\end{bmatrix},
\bm{x}=\begin{bmatrix}
x_1 \\
x_2 \\
\vdots \\
x_n 
\end{bmatrix}
\end{equation*}\par
则二次型可以表示为:
\begin{equation*}
f(\bm{x})=\bm{x}^{T}\bm{A}\bm{x}
\end{equation*}\par
必须强调的是, 这里的$ \bm{A} $是一个对称矩阵.
\subsection{线性变换}
对于$ n $元二次型$ f(x_{1}, x_{2},... ,x_{n}) $, 若令
\begin{equation*}
\left\{
\begin{aligned}
& x_{1} = c_{11}y_{1}+c_{12}y_{2}+\dots +c_{1n}y_{n}, \\ 
& x_{2} = c_{21}y_{1}+c_{22}y_{2}+\dots +c_{2n}y_{n}, \\
& \dots \\
& x_{n} = c_{n1}y_{1}+c_{n2}y_{2}+\dots +c_{nn}y_{n},
\end{aligned}
\right.
\end{equation*}\par
记$ \bm{x}=\begin{bmatrix}
x_1 \\
x_2 \\
\vdots \\
x_n 
\end{bmatrix}, \bm{C}=\begin{bmatrix}
c_{11} & c_{12} & \dots & c_{1n} \\
c_{21} & c_{22} & \dots & c_{2n} \\
\vdots & \vdots &  & \vdots \\
c_{n1} & c_{n2} & \dots & c_{nn} 
\end{bmatrix}, \bm{y}=\begin{bmatrix}
y_1 \\
y_2 \\
\vdots \\
y_n 
\end{bmatrix}$\par \vspace{1em}
则上式可以写为$ \bm{x}=\bm{C}\bm{y} $. 上式成为从$ y_{1}, y_{2},... ,y_{n} $到$ x_{1}, x_{2},... ,x_{n} $的线性变换. 如果$ \bm{C} $可逆, 则称为可逆线性变换.\par
如果$ f(\bm{x})=\bm{x}^{T}\bm{A}\bm{x} $, 令$ \bm{x}=\bm{C}\bm{y} $, 则有$ f(\bm{x})=(\bm{C}\bm{y})^{T}\bm{A}(\bm{C}\bm{y})=\bm{y}^{T}(\bm{C}^{T}\bm{A}\bm{C})\bm{y}. $\par
记$ \bm{B}=\bm{C}^{T}\bm{A}\bm{C} $, 则有$ f(\bm{x})=\bm{y}^{T}\bm{B}\bm{y})=g(\bm{y}) $. 至此我们通过线性变换得到了一个新的二次型.
\subsection{矩阵合同}
\subsubsection{定义}
设$ \bm{A}, \bm{B} $为$ n $阶矩阵, 若存在可逆矩阵$ \bm{C} $, 使得:
\begin{equation*}
\bm{C}^{T}\bm{A}\bm{C}=\bm{B}
\end{equation*}\par
则称$ \bm{A} $和$ \bm{B} $合同, 记作$ \bm{A}\simeq \bm{B} $. 此时称$ f(\bm{x}) $与$ g(\bm{x}) $为合同二次型.\par
所谓合同, 就是指同一个二次型在可逆线性变换下的两个不同状态的联系.
\subsubsection{性质}
\begin{enumerate}
\item 反身性: $ \bm{A}\simeq \bm{A} $
\item 对称性: $ \bm{A}\simeq \bm{B}\Rightarrow \bm{B}\simeq \bm{A} $
\item 传递性: $ \bm{A}\simeq \bm{B}, \bm{B}\simeq \bm{C}\Rightarrow \bm{A}\simeq \bm{C} $
\end{enumerate}
\subsection{标准形/规范形}
\subsubsection{定义}
若二次型中只含有平方项, 没有交叉项, 形如
\begin{equation*}
d_{1}x_{1}^{2}+d_{2}x_{2}^{2}+... +d_{n}x_{n}^{2}
\end{equation*}\par
的二次型称为标准形.\par
若标准形中, 系数$ d_{i} $仅为$ 1, -1, 0 $的二次型称为规范形.
\subsubsection{求法}
我们的目标是使得$ \bm{B} $矩阵是一个对角矩阵, 即只有主对角线有元素, 才可以得到标准型. 有两种方法:
\begin{enumerate}
\item 任何二次型可以通过配方法(作可逆线性变换)化为标准形和规范形, 它求得的对角矩阵(标准形)形式如下(不一定是特征值$ \lambda $):
\begin{equation*}
\bm{\Lambda}=\begin{bmatrix}
d_1 &  &  &  \\
& d_2 &  &  \\
&  & \ddots &  \\
&  &  & d_n 
\end{bmatrix}
\end{equation*}\par
此外, 它还可以转化成规范形:
\begin{equation*}
\bm{\Lambda}=\begin{bmatrix}
1 &  &  &  &  &  &  &  &  \\
& \ddots &  &  &  &  &  &  &  \\
&  & 1 &  &  &  &  &  &  \\
&  &  & -1 &  &  &  &  &  \\
&  &  &  & \ddots &  &  &  &  \\
&  &  &  &  & -1 &  &  &  \\
&  &  &  &  &  & 0 &  &  \\
&  &  &  &  &  &  & \ddots &  \\
&  &  &  &  &  &  &  & 0 
\end{bmatrix}
\end{equation*}
\item 任何二次型可以通过正交变换化成标准形(见\ref{ref:求标准形}), 它求得的对角矩阵(标准形)形式如下(特征值不一定是$ 0, 1, -1 $):
\begin{equation*}
\bm{\Lambda}=\begin{bmatrix}
\lambda_1 &  &  &  \\
& \lambda_2 &  &  \\
&  & \ddots &  \\
&  &  & \lambda_n
\end{bmatrix}
\end{equation*}
\end{enumerate}
\subsection{惯性定理}
\subsubsection{定义}
无论选取什么样的线性变换(配方还是正交合同变换), 将二次型化为标准形或者规范形, 其正项系数个数$ p $, 负项个数$ q $都是不变的, $ p $称为正惯性指数, $ q $称为负惯性指数.
\subsubsection{性质}
\begin{enumerate}
\item 若二次型的秩为$ r $, 则$ r=p+q $, 可逆线性变换不改变正/负惯性指数
\item 两个二次型(或者实对称矩阵)合同的条件是有相同的正/负惯性指数
\end{enumerate}
\subsection{正定二次型及其判别}
\subsubsection{定义}
$ n $元二次型$ f(x_{1}, x_{2},... ,x_{n})=\bm{x}^{T}\bm{A}\bm{x} $, 若对于任意的$ \bm{x}=[x_{1},x_{2},... ,x_{n}]^{T}\neq \bm{0} $均有二次型大于0, 即$ \bm{x}^{T}\bm{A}\bm{x}>0 $, 则称$ f $为正定二次型, $ \bm{A} $为正定矩阵.
\subsubsection{条件}
\begin{enumerate}
\item 充要条件: \par $ \begin{aligned}
n\text{元二次型正定} & \Leftrightarrow \text{对于任意}\bm{x}\neq \bm{0}, \text{有}\bm{x}^{T}\bm{A}\bm{x}>0 \\
& \Leftrightarrow f\text{的正惯性指数} p=n\text{(所有的系数全正)} \\
& \Leftrightarrow \text{存在可逆矩阵} \bm{D}, \text{使} \bm{A}=\bm{D}^{T}\bm{D} \\
& \Leftrightarrow \bm{A}\simeq \bm{E} \\
& \Leftrightarrow \bm{A}\text{的特征值} \lambda_{i}>0(i=1, 2,... ,n) \\
& \Leftrightarrow \bm{A}\text{的全部顺序主子式均大于0(左上角行列式)}
\end{aligned} $
\item 必要条件: \par $ \begin{aligned}
n\text{元二次型正定} & \Leftarrow a_{ii}>0(i=1, 2,... ,n) \\
& \Leftarrow \left|\bm{A}\right|>0 
\end{aligned} $
\end{enumerate}
\section{习题}
\subsection{标准形/规范形}
\subsubsection{用正交变换法化二次型为标准形}
\begin{enumerate}\label{ref:求标准形}
\item 写出二次型矩阵$ \bm{A} $
\item 求$ \bm{A} $的特征值$ \lambda $和特征向量$ \bm{\xi} $
\item 将$ \bm{\xi_{1}},..., \bm{\xi_{n}} $通过正交化/单位化成正交矩阵$ \bm{Q}=(\bm{\eta_{1}},..., \bm{\eta_{n}}) $
\item 令$ \bm{x}=\bm{Q}\bm{y}\Rightarrow f(\bm{x})=\bm{x}^{T}\bm{A}\bm{x}=(\bm{Q}\bm{y})^{T}\bm{A}\bm{Q}\bm{y}=\bm{y}^{T}\bm{Q}^{-1}\bm{A}\bm{Q}\bm{y}=\bm{y}^{T}\bm{\lambda}\bm{y}\Rightarrow f(y_{1},..., y_{n})=\lambda_{1}y_{1}^{2}+... +\lambda_{n}y_{n}^{2} $ 
\end{enumerate}
\paragraph{注意} 正交变换只能化二次型为标准形, 不能化为规范形(除非特征值都是$ 0, 1, -1 $)\par
