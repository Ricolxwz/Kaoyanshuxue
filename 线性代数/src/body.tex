\chapter{特征值和特征向量}
\section{基础知识}
\subsection{特征值和特征向量}
\subsubsection{定义}
设$ \bm{A} $为$ n $阶矩阵, $ \lambda $是一个数, 若存在一个非零的$ n $维向量$ \bm{\xi} $, 使得$ \bm{A}\bm{\xi}=\lambda \bm{\xi} $, 则称$ \bm{\xi} $为$ \bm{A} $的特征向量, $ \lambda $为$ \bm{A} $的特征值. 
\par 上式可以化简成$ \left| \lambda \bm{E}-\bm{A}\right|\bm{\xi}=\bm{0} $, $ \left| \lambda \bm{E}-\bm{A}\right| $被称为特征多项式, $ \lambda \bm{E}-\bm{A} $称为特征矩阵.
\subsubsection{性质}
\begin{enumerate}
	\item 特征值的性质
	\begin{enumerate}
		\item $ \sum_{i=1}^{n}\lambda_{i}=\sum_{i=1}^{n}a_{ii}=tr(\bm{A}) $
		\item $ \prod_{i=1}^{n}\lambda_{i}=\left| \bm{A}\right| $
	\end{enumerate}
	\item 特征向量的性质
	\begin{enumerate}
		\item $ k $重特征值$ \Lambda $至多只有$ k $个线性无关的向量
		\item 若$ \bm{\xi_1} $, $ \bm{\xi_2} $是$ \bm{A} $的属于不同特征值的特征的特征向量, 则$ \bm{\xi_1}, \bm{\xi_2} $线性无关
		\item 若$ \bm{\xi_1}, \bm{\xi_2} $是$ \bm{A} $的属于同一特征值$ \lambda $的特征向量, 则$ k_1\bm{\xi_1} + k_1\bm{\xi_2} $仍然是$ \bm{A} $的属于特征值$ \lambda $的特征向量
	\end{enumerate}
\end{enumerate}
\subsection{矩阵的相似}
\subsubsection{定义}
设$ \bm{A} $和$ \bm{B} $为两个$ n $阶方阵, 若存在$ n $阶可逆矩阵$ \bm{P} $, 使得$ \bm{P}^{-1}\bm{A}\bm{P}=\bm{B} $成立, 则称$ \bm{A} $相似于$ \bm{B} $, 记成$ \bm{A}\sim \bm{B} $.
\subsubsection{性质}
\begin{enumerate}
	\item \begin{itemize}
		\item 反身性: $ \bm{A}\sim \bm{A} $
		\item 对称性: $ \bm{A}\sim \bm{B}\Rightarrow \bm{B}\sim \bm{A} $
		\item 传递性: $ \bm{A}\sim \bm{B}, \bm{B}\sim \bm{C}\Rightarrow \bm{A}\sim \bm{C} $
	\end{itemize}
	\item 若$ \bm{A}\sim \bm{B} $, 则有\begin{itemize}
		\item $ r(\bm{A})=r(\bm{B}) $
		\item $ \left|\bm{A}\right|=\left|\bm{B}\right| $
		\item $ \bm{A}, \bm{B} $具有相同的特征值
		\item $ \bm{A}, \bm{B} $特征多项式的值相同
	\end{itemize}
	\item 若$ \bm{A}\sim \bm{B} $, 则有\begin{itemize}
		\item $ f(\bm{A})\sim f(\bm{B}) $
		\item $ \bm{A}^{T}\sim \bm{B}^{T} $
		\item $ \bm{A} $可逆, $ \bm{A}^{*}\sim \bm{B}^{*} $
		\item $ \bm{A} $可逆, $ \bm{A}^{-1}\sim \bm{B}^{-1} $
	\end{itemize}
\end{enumerate}
\subsection{矩阵的相似对角化}
\subsubsection{定义}
设$ n $阶矩阵$ \bm{A} $, 存在$ n $阶可逆矩阵$ \bm{P} $, 使得$ \bm{P}^{-1}\bm{A}\bm{P}=\bm{\Lambda} $, 则$ \bm{A}\sim \bm{\Lambda} $, $ \bm{\Lambda} $是$ \bm{A} $的相似标准形.\par 
\[ \bm{P}=\left[\bm{\xi_1}, \bm{\xi_2},... \bm{\xi_n}\right], \bm{\Lambda} =\begin{bmatrix}
	\lambda_1 &  &  &  \\
	& \lambda_2 &  &  \\
	&  & \ddots &  \\
	&  &  & \lambda_n 
\end{bmatrix}
\]
\subsubsection{条件}
\begin{enumerate}
	\item $ n $阶矩阵$ \bm{A} $可以相似对角化$ \Leftrightarrow $$ \bm{A} $有$ n $个线性无关的特征向量($ \left| \bm{P}\right| = 0$)
	\item $ n $阶矩阵$ \bm{A} $可以相似对角化$ \Leftrightarrow $$ \bm{A} $对应于每个$ k_i $重特征值都有$ k_i $个线性无关的特征向量($ n $重特征值对应的解空间是$ n $维)
	\item $ n $阶矩阵$ \bm{A} $有$ n $个不同特征值$ \Rightarrow $$ \bm{A} $可以相似对角化(由特征向量的性质3可以推出)
	\item $ n $阶矩阵$ \bm{A} $为实对称矩阵$ \Rightarrow $$ \bm{A} $可以相似对角化
\end{enumerate}\par
上述总共两个充要条件, 两个充分条件.
\subsection{实对称矩阵}
\subsubsection{定义}
若$ \bm{A}^T = \bm{A} $, 则$ \bm{A} $为是对称矩阵, 如果在此基础上$ \bm{A} $的元素都是实数, 则$ \bm{A} $是实对称矩阵.
\subsubsection{性质}
\begin{enumerate}\label{ref:实对称矩阵}
	\item 实对称矩阵$ \bm{A} $的属于不同特征值的特征向量相互正交
	\item 实对称矩阵$ \bm{A} $必相似于对角矩阵, 必有可逆矩阵$ \bm{P}=\left[ \bm{\xi_{1}}, \bm{\xi_{2}},... ,\bm{\xi_{n}}\right] $, 使得$ \bm{P}^{-1}\bm{A}\bm{P}=\bm{\Lambda} $. 且存在正交矩阵$ \bm{Q} $, 使得$ \bm{Q}^{-1}\bm{A}\bm{Q}=\bm{Q}^{T}\bm{A}\bm{Q}=\bm{\Lambda}$,  
\end{enumerate}
\section{习题}
\subsection{特征值和特征向量}
\subsubsection{求具体型矩阵的特征值和特征向量}
\begin{enumerate}
	\item 用特征方程$ \left|\lambda \bm{E}-\bm{A}\right|=0 $求出$ \lambda $, 可以使用试根法对$ \lambda $的高次方程进行求解
	\item 用求得的$ \lambda $解齐次线性方程组$ (\lambda \bm{E}-\bm{A})\bm{\xi}=\bm{0} $, 求出特征向量
\end{enumerate}
\subsubsection{求解抽象型矩阵的特征值和特征向量}
\begin{table}[H]
	\centering
	\begin{tabular}{|c|ccccccc|}
		\hline
		\textbf{矩阵} & $ \bm{A} $ & $ k\bm{A} $ & $ \bm{A}^{k} $ & $ f(\bm{A}) $ & $ \bm{A}^{-1} $ & $ \bm{A}^{*} $ & $ \bm{P}^{-1}\bm{A}\bm{P} $ \\ \hline
		\textbf{特征值} & $ \lambda $ & $ k\lambda $ & $ \lambda^{k} $ & $ f(\lambda) $ & $ \lambda^{-1} $ & $ \frac{\left|\bm{A}\right|}{\lambda} $ & $ \lambda $\\ \hline
		\textbf{特征向量} & $ \bm{\xi} $ & $ \bm{\xi} $ & $ \bm{\xi} $ & $ \bm{\xi} $ & $ \bm{\xi} $ & $ \bm{\xi} $ & $ \bm{P}^{-1}\bm{\xi} $\\ 
		\hline
	\end{tabular}
\end{table}\par
$ f(x) $为多项式, 若矩阵$ \bm{A} $满足$ f(\bm{A})=\bm{0}\Rightarrow f(\lambda)=0 $.
\subsection{实对称矩阵}
\subsubsection{求正交矩阵Q}
\begin{enumerate}
	\item 求$ \bm{A} $的$ \lambda $与$ \bm{\xi} $
	\item $ \bm{\xi_1}, \bm{\xi_{2}},... ,\bm{\xi_{n}} $施密特正交化, 单位化至$ \bm{\eta_{1}}, \bm{\eta_{2}},... ,\bm{\eta_{n}} $
	\item 令$ Q=(\bm{\eta_1}, \bm{\eta_2},... ,\bm{\eta_{n}}) $
\end{enumerate}
\par 不同的特征值$ \lambda_{i} $对应的特征矩阵$ \bm{\xi_{i}} $之间是正交的.
\par 施密特正交化:$ \beta_{1}=\alpha_{1}, 
\beta_{2}=\alpha_{2}-\frac{(\alpha_{2}, \beta_{1})}{(\beta_{1}, \beta_{1})}\beta_{1} $.
\par 单位化: $ \eta_{1}=\frac{\beta_{1}}{||\beta_{1}||} $.
\paragraph{总结} 
\begin{enumerate}
	\item 普通矩阵$ \bm{A} $
	\begin{enumerate}
		\item $ \lambda_{1}\neq \lambda_{2}\Rightarrow \xi_{1}, \xi_{2}$无关
		\item $ \lambda_{1}= \lambda_{2}\Rightarrow \xi_{1}, \xi_{2}$
		\begin{enumerate}
			\item $\xi_{1}, \xi_{2}$无关
			\item $\xi_{1}, \xi_{2}$相关
		\end{enumerate}
	\end{enumerate}
	\item 实对称矩阵$ \bm{A} $
	\begin{enumerate}
		\item $ \lambda_{1}\neq \lambda_{2}\Rightarrow \xi_{1}\perp \xi_{2}$\quad $\xi_{1}, \xi_{2}$无关
		\item $ \lambda_{1}= \lambda_{2}\Rightarrow$ 
		\begin{enumerate}
			\item $\xi_{1}\perp \xi_{2}$\quad $\xi_{1}, \xi_{2}$无关
			\item $\xi_{1}$不垂直于$\xi_{2}$\quad $\xi_{1}, \xi_{2}$无关
		\end{enumerate}
	\end{enumerate}
\end{enumerate}
\chapter{二次型}
\section{基础知识}
\subsection{二次型}
\subsubsection{定义}
$ n $元变量$ x_{1}, x_{2},... ,x_{n} $的二次齐次多项式称为$ n $元二次型, 简称二次型.\par
二次型可以表示为$ \sum_{i=1}^{n}\sum_{j=1}^{n}a_{ij}x_{i}x_{j} $, 由此可以得出二次型的矩阵表达式, 令:
\begin{equation*}
	\bm{A}=\begin{bmatrix}
		a_{11} & a_{12} & \dots & a_{1n} \\
		a_{21} & a_{22} & \dots & a_{2n} \\
		\vdots & \vdots &  & \vdots \\
		a_{n1} & a_{n2} & \dots & a_{nn} 
	\end{bmatrix},
	\bm{x}=\begin{bmatrix}
		x_1 \\
		x_2 \\
		\vdots \\
		x_n 
	\end{bmatrix}
\end{equation*}\par
则二次型可以表示为:
\begin{equation*}
	f(\bm{x})=\bm{x}^{T}\bm{A}\bm{x}
\end{equation*}\par
必须强调的是, 这里的$ \bm{A} $是一个对称矩阵.
\subsection{线性变换}
对于$ n $元二次型$ f(x_{1}, x_{2},... ,x_{n}) $, 若令
\begin{equation*}
	\left\{
	\begin{aligned}
		& x_{1} = c_{11}y_{1}+c_{12}y_{2}+\dots +c_{1n}y_{n}, \\ 
		& x_{2} = c_{21}y_{1}+c_{22}y_{2}+\dots +c_{2n}y_{n}, \\
		& \dots \\
		& x_{n} = c_{n1}y_{1}+c_{n2}y_{2}+\dots +c_{nn}y_{n},
	\end{aligned}
	\right.
\end{equation*}\par
记$ \bm{x}=\begin{bmatrix}
	x_1 \\
	x_2 \\
	\vdots \\
	x_n 
\end{bmatrix}, \bm{C}=\begin{bmatrix}
	c_{11} & c_{12} & \dots & c_{1n} \\
	c_{21} & c_{22} & \dots & c_{2n} \\
	\vdots & \vdots &  & \vdots \\
	c_{n1} & c_{n2} & \dots & c_{nn} 
\end{bmatrix}, \bm{y}=\begin{bmatrix}
	y_1 \\
	y_2 \\
	\vdots \\
	y_n 
\end{bmatrix}$\par \vspace{1em}
则上式可以写为$ \bm{x}=\bm{C}\bm{y} $. 上式成为从$ y_{1}, y_{2},... ,y_{n} $到$ x_{1}, x_{2},... ,x_{n} $的线性变换. 如果$ \bm{C} $可逆, 则称为可逆线性变换.\par
如果$ f(\bm{x})=\bm{x}^{T}\bm{A}\bm{x} $, 令$ \bm{x}=\bm{C}\bm{y} $, 则有$ f(\bm{x})=(\bm{C}\bm{y})^{T}\bm{A}(\bm{C}\bm{y})=\bm{y}^{T}(\bm{C}^{T}\bm{A}\bm{C})\bm{y}. $\par
记$ \bm{B}=\bm{C}^{T}\bm{A}\bm{C} $, 则有$ f(\bm{x})=\bm{y}^{T}\bm{B}\bm{y})=g(\bm{y}) $. 至此我们通过线性变换得到了一个新的二次型.
\subsection{矩阵合同}
\subsubsection{定义}
设$ \bm{A}, \bm{B} $为$ n $阶矩阵, 若存在可逆矩阵$ \bm{C} $, 使得:
\begin{equation*}
	\bm{C}^{T}\bm{A}\bm{C}=\bm{B}
\end{equation*}\par
则称$ \bm{A} $和$ \bm{B} $合同, 记作$ \bm{A}\simeq \bm{B} $. 此时称$ f(\bm{x}) $与$ g(\bm{x}) $为合同二次型.\par
所谓合同, 就是指同一个二次型在可逆线性变换下的两个不同状态的联系.
\subsubsection{性质}
\begin{enumerate}
	\item 反身性: $ \bm{A}\simeq \bm{A} $
	\item 对称性: $ \bm{A}\simeq \bm{B}\Rightarrow \bm{B}\simeq \bm{A} $
	\item 传递性: $ \bm{A}\simeq \bm{B}, \bm{B}\simeq \bm{C}\Rightarrow \bm{A}\simeq \bm{C} $
\end{enumerate}
\subsection{标准形/规范形}
\subsubsection{定义}
若二次型中只含有平方项, 没有交叉项, 形如
\begin{equation*}
	d_{1}x_{1}^{2}+d_{2}x_{2}^{2}+... +d_{n}x_{n}^{2}
\end{equation*}\par
的二次型称为标准形.\par
若标准形中, 系数$ d_{i} $仅为$ 1, -1, 0 $的二次型称为规范形.
\subsubsection{求法}
我们的目标是使得$ \bm{B} $矩阵是一个对角矩阵, 即只有主对角线有元素, 才可以得到标准型. 有两种方法:
\begin{enumerate}
	\item 任何二次型可以通过配方法(作可逆线性变换)化为标准形和规范形, 它求得的对角矩阵(标准形)形式如下(不一定是特征值$ \lambda $):
	\begin{equation*}
		\bm{\Lambda}=\begin{bmatrix}
			d_1 &  &  &  \\
			& d_2 &  &  \\
			&  & \ddots &  \\
			&  &  & d_n 
		\end{bmatrix}
	\end{equation*}\par
	此外, 它还可以转化成规范形:
	\begin{equation*}
		\bm{\Lambda}=\begin{bmatrix}
			1 &  &  &  &  &  &  &  &  \\
			& \ddots &  &  &  &  &  &  &  \\
			&  & 1 &  &  &  &  &  &  \\
			&  &  & -1 &  &  &  &  &  \\
			&  &  &  & \ddots &  &  &  &  \\
			&  &  &  &  & -1 &  &  &  \\
			&  &  &  &  &  & 0 &  &  \\
			&  &  &  &  &  &  & \ddots &  \\
			&  &  &  &  &  &  &  & 0 
		\end{bmatrix}
	\end{equation*}
	\item 任何二次型可以通过正交变换化成标准形(见\ref{ref:求标准形}), 它求得的对角矩阵(标准形)形式如下(特征值不一定是$ 0, 1, -1 $):
	\begin{equation*}
		\bm{\Lambda}=\begin{bmatrix}
			\lambda_1 &  &  &  \\
			& \lambda_2 &  &  \\
			&  & \ddots &  \\
			&  &  & \lambda_n
		\end{bmatrix}
	\end{equation*}
\end{enumerate}
\subsection{惯性定理}
\subsubsection{定义}
无论选取什么样的线性变换(配方还是正交合同变换), 将二次型化为标准形或者规范形, 其正项系数个数$ p $, 负项个数$ q $都是不变的, $ p $称为正惯性指数, $ q $称为负惯性指数.
\subsubsection{性质}
\begin{enumerate}
	\item 若二次型的秩为$ r $, 则$ r=p+q $, 可逆线性变换不改变正/负惯性指数
	\item 两个二次型(或者实对称矩阵)合同的条件是有相同的正/负惯性指数
\end{enumerate}
\subsection{正定二次型及其判别}
\subsubsection{定义}
$ n $元二次型$ f(x_{1}, x_{2},... ,x_{n})=\bm{x}^{T}\bm{A}\bm{x} $, 若对于任意的$ \bm{x}=[x_{1},x_{2},... ,x_{n}]^{T}\neq \bm{0} $均有二次型大于0, 即$ \bm{x}^{T}\bm{A}\bm{x}>0 $, 则称$ f $为正定二次型, $ \bm{A} $为正定矩阵.
\subsubsection{条件}
\begin{enumerate}
	\item 充要条件: \par $ \begin{aligned}
		n\text{元二次型正定} & \Leftrightarrow \text{对于任意}\bm{x}\neq \bm{0}, \text{有}\bm{x}^{T}\bm{A}\bm{x}>0 \\
		& \Leftrightarrow f\text{的正惯性指数} p=n\text{(所有的系数全正)} \\
		& \Leftrightarrow \text{存在可逆矩阵} \bm{D}, \text{使} \bm{A}=\bm{D}^{T}\bm{D} \\
		& \Leftrightarrow \bm{A}\simeq \bm{E} \\
		& \Leftrightarrow \bm{A}\text{的特征值} \lambda_{i}>0(i=1, 2,... ,n) \\
		& \Leftrightarrow \bm{A}\text{的全部顺序主子式均大于0(左上角行列式)}
	\end{aligned} $
	\item 必要条件: \par $ \begin{aligned}
		n\text{元二次型正定} & \Leftarrow a_{ii}>0(i=1, 2,... ,n) \\
		& \Leftarrow \left|\bm{A}\right|>0 
	\end{aligned} $
\end{enumerate}
\section{习题}
\subsection{标准形/规范形}
\subsubsection{用正交变换法化二次型为标准形}
\begin{enumerate}\label{ref:求标准形}
	\item 写出二次型矩阵$ \bm{A} $
	\item 求$ \bm{A} $的特征值$ \lambda $和特征向量$ \bm{\xi} $
	\item 将$ \bm{\xi_{1}},..., \bm{\xi_{n}} $通过正交化/单位化成正交矩阵$ \bm{Q}=(\bm{\eta_{1}},..., \bm{\eta_{n}}) $
	\item 令$ \bm{x}=\bm{Q}\bm{y}\Rightarrow f(\bm{x})=\bm{x}^{T}\bm{A}\bm{x}=(\bm{Q}\bm{y})^{T}\bm{A}\bm{Q}\bm{y}=\bm{y}^{T}\bm{Q}^{-1}\bm{A}\bm{Q}\bm{y}=\bm{y}^{T}\bm{\lambda}\bm{y}\Rightarrow f(y_{1},..., y_{n})=\lambda_{1}y_{1}^{2}+... +\lambda_{n}y_{n}^{2} $ 
\end{enumerate}
\paragraph{注意} 正交变换只能化二次型为标准形, 不能化为规范形(除非特征值都是$ 0, 1, -1 $)